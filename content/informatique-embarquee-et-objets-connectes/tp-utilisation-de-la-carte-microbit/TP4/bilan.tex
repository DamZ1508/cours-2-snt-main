%%%%%%%%%%%%%%%%%%%%%%%%%%%%%%%%%%%%%%%%%%%%%%%%%%%%%%%%%%%%%%%%%%%%%%%%%%%%%%%%
% Copyright 2023 Louis Paternault --- http://snt.ababsurdo.fr
%
% Publié sous licence Creative Commons Attribution-ShareAlike 4.0 International (CC BY-SA 4.0)
% http://creativecommons.org/licenses/by-sa/4.0/deed.fr
%%%%%%%%%%%%%%%%%%%%%%%%%%%%%%%%%%%%%%%%%%%%%%%%%%%%%%%%%%%%%%%%%%%%%%%%%%%%%%%%

% Pour compiler :
%$ lualatex $basename

\documentclass[12pt]{article}

\usepackage{textcomp}
\usepackage{fontspec}
\usepackage{polyglossia}
\setmainlanguage{french}
\usepackage{calc}
\usepackage[
  paperwidth={21cm},
  paperheight={29.7cm/3},
  includehead,
  headsep=3mm,
  margin=7mm]{geometry}

\usepackage{hyperref}
\hypersetup{
  unicode=true,
  urlcolor=cyan,
  pdfauthor={Louis Paternault},
  pdfproducer={© Louis Paternault — CC-BY-SA-4.0 — http://snt.ababsurdo.fr},
  hidelinks,
}

\usepackage{fancyhdr}
\fancyhead[L]{\textsc{SNT > Informatique embarquée > TP4}}
\fancyfoot{}
\pagestyle{fancy}

\usepackage{xcolor}
\usepackage{tabularray}
\UseTblrLibrary{booktabs}

\setlength{\parindent}{0pt}

% Merci gernot
% https://tex.stackexchange.com/a/332124
\newcommand\fillin[1][3cm]{\makebox[#1]{\dotfill}}

\begin{document}

\begin{tblr}{XX}
\textbf{Votre nom : \dotfill}
&
\textbf{Nom de votre partenaire : \dotfill}
\\
\end{tblr}

\vfill

\begin{tblr}{
    colspec={Xr},
    row{even} = {lightgray},
    row{1, 4} = {font=\bfseries\large},
    cell{2-3}{1} = {
      halign=l,
      valign=h,
    },
    cell{2-3}{2} = {
      halign=r,
      valign=h,
    },
  }
  \toprule
  Questions & …/2 \\
  {
    Écrire la commande permettant de choisir au hasard et d'afficher un élément de la liste \texttt{LISTE}. \\[4mm]
    \texttt{\fillin[5cm]}\\[4mm]
} & …/1 \\
{
  Écrire la condition permettant de tester si la carte est secouée.\\[4mm]
  \texttt{if \fillin[4cm]:}\\[4mm]
} & …/1 \\
  \midrule
  Programme : Choixpeau & …/2 \\
  Une des quatre maisons défile au hasard quand la carte est secouée. & …/1 \\
  Cela fonctionne plusieurs fois à la suite, en affichant un point d'interrogation en attendant. & …/1 \\
  \bottomrule
\end{tblr}

\end{document}
