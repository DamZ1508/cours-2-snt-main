%%%%%%%%%%%%%%%%%%%%%%%%%%%%%%%%%%%%%%%%%%%%%%%%%%%%%%%%%%%%%%%%%%%%%%%%%%%%%%%%
% Copyright 2023 Louis Paternault --- http://snt.ababsurdo.fr
%
% Publié sous licence Creative Commons Attribution-ShareAlike 4.0 International (CC BY-SA 4.0)
% http://creativecommons.org/licenses/by-sa/4.0/deed.fr
%%%%%%%%%%%%%%%%%%%%%%%%%%%%%%%%%%%%%%%%%%%%%%%%%%%%%%%%%%%%%%%%%%%%%%%%%%%%%%%%

% Pour compiler :
%$ lualatex $basename

\documentclass[12pt]{article}

\usepackage{textcomp}
\usepackage{fontspec}
\usepackage{polyglossia}
\setmainlanguage{french}
\usepackage{hyperref}
\usepackage[
  a5paper,
  includehead,
  landscape,
  headsep=3mm,
  margin=10mm]{geometry}

\hypersetup{
  unicode=true,
  urlcolor=cyan,
  pdfauthor={Louis Paternault},
  pdfproducer={© Louis Paternault — CC-BY-SA-4.0 — http://snt.ababsurdo.fr},
  hidelinks,
}

\usepackage{fancyhdr}
\fancyhead[L]{\textsc{SNT > Informatique embarquée > TP5}}
\fancyfoot{}
\pagestyle{fancy}

\usepackage{xcolor}
\usepackage{tabularray}
\UseTblrLibrary{booktabs}

\setlength{\parindent}{0pt}

\begin{document}

\begin{tblr}{XX}
\textbf{Votre nom : \dotfill}
&
\textbf{Nom de votre partenaire : \dotfill}
\\
\end{tblr}

\vfill

\begin{tblr}{
    colspec={Xr},
    row{even} = {lightgray},
    row{1, 5} = {font=\bfseries\large},
    row{2-4} = {ht=15mm},
    cell{2-4}{1} = {
      halign=l,
      valign=h,
    },
    cell{2-4}{2} = {
      halign=r,
      valign=h,
    },
  }
  \toprule
  Questions & …/3 \\
  Écrire la commande permettant de configurer la radio pour utiliser le canal 42. & …/1 \\
  Écrire la commande permettant d'envoyer le message \texttt{"Bonjour"}. & …/1 \\
  Écrire la commande permettant de recevoir un message, et de tester si ce message est \texttt{"Bonjour"} (une ou deux lignes). & …/1 \\[2cm]
  \midrule
  Programme : Bracelet, Clignotant, Station météo & …/2 \\
  Le programme fonctionne comme attendu. \\[2cm]
  \bottomrule
\end{tblr}

\end{document}
