%%%%%%%%%%%%%%%%%%%%%%%%%%%%%%%%%%%%%%%%%%%%%%%%%%%%%%%%%%%%%%%%%%%%%%%%%%%%%%%%
% Copyright 2021 Louis Paternault --- http://ababsurdo.fr
%
% Publié sous licence Creative Commons Attribution-ShareAlike 4.0 International (CC BY-SA 4.0)
% http://creativecommons.org/licenses/by-sa/4.0/deed.fr
%%%%%%%%%%%%%%%%%%%%%%%%%%%%%%%%%%%%%%%%%%%%%%%%%%%%%%%%%%%%%%%%%%%%%%%%%%%%%%%%

% Pour compiler :
%$ lualatex $basename

\documentclass[12pt]{article}

\usepackage{2122-pablo}
\usepackage{2122-pablo-paternault}
\usepackage{2122-pablo-math}

\usepackage[
	includehead,
	a5paper,
	margin=1cm,
]{geometry}
\usepackage{2122-pablo-header}
\fancyhead[C]{\textsc{Ex. 9 p. 75 --- Correction}}

\begin{document}

\begin{exercice}[Exercice 9 page 75]~
	\begin{enumerate}
		\item Le graphe est composé de neuf sommets (les \enquote{cercles}) et huit arêtes (les \enquote{traits}).
		\item Calculons l'écartement de chacun des sommets. Par exemple, les sommets les plus éloignés de 1 sont 7, 8 et 9, à une distance de 3, donc l'écartement de 1 est 3 ; le sommet le plus éloigné de 7 est 9, à une distance de 6, donc l'écartement de 9 est 6.
\begin{center}\begin{tabular}{r*{9}{c}}
\toprule
	Sommet & 1&2&3&4&5&6&7&8&9 \\
\midrule
	Écartement & 3&4&4&5&5&5&6&6&6 \\
\bottomrule
\end{tabular}\end{center}
\item Le centre du graphe est le sommet qui a le plus petit écartement : c'est donc 1.
	\end{enumerate}
\end{exercice}

\end{document}
