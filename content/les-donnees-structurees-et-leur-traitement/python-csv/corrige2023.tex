%%%%%%%%%%%%%%%%%%%%%%%%%%%%%%%%%%%%%%%%%%%%%%%%%%%%%%%%%%%%%%%%%%%%%%%%%%%%%%%%
% Copyright 2023 Louis Paternault --- http://ababsurdo.fr
%
% Reproduction, diffusion, et modification interdites, sauf exceptions prévues par la loi.
%%%%%%%%%%%%%%%%%%%%%%%%%%%%%%%%%%%%%%%%%%%%%%%%%%%%%%%%%%%%%%%%%%%%%%%%%%%%%%%%

% Pour compiler :
%$ lualatex $basename

\documentclass[12pt]{article}

\usepackage{2223-pablo}
\usepackage[python]{2223-pablo-listings}

\usepackage[
	a4paper,
	includehead,
	margin=10mm,
	headsep=3mm,
]{geometry}
\usepackage{2223-pablo-header}
\fancyhead[L]{\textsc{SNT > Données > Python et format CSV}}
\fancyhead[R]{\textsc{Corrigé}}

\def\thesubsection{\arabic{section}.\alph{subsection}}

\begin{document}

\setcounter{section}{2}
\section{Exemple pas à pas}

\setcounter{subsection}{2}
\subsection{Parcours du tableau}

\begin{enumerate}
	\item  En 2021, dans le département d'Ille et Villaine (35), 5 filles sont nées avec le prénom Zya.
		\stepcounter{enumi}
	\item \lstinline{recherche = prenoms.loc[3784670, "preusuel"]}
	\item
		\begin{lstlisting}
		import pandas
		prenoms = pandas.read_csv("dpt2021.csv", sep=";")
		recherche = prenoms.loc[:, "preusuel"]
		print(recherche)
		\end{lstlisting}
\end{enumerate}
\subsection{Recherche dans le tableau}
\begin{enumerate}
		\setcounter{enumi}{4}
	\item
		\begin{enumerate}
			\item En 2019 en Isère, le prénom le plus donné a été Léo (88 fois).
			\item En 2019 en Isère, le prénom le plus donné aux filles a été Emma (69 fois).
			\item En 2019 en Isère, 78 garçons ont été nommés Gabriel.
		\end{enumerate}
	\item
		\begin{enumerate}
			\item \emph{Ça dépend…}
			\item \emph{Ça dépend…}
			\item \emph{Ça dépend…}
		\end{enumerate}
\end{enumerate}

\section{Un peu plus d'autonomie}

\begin{enumerate}
		\setcounter{enumi}{2}
	\item
		\begin{enumerate}
			\item On a simplement changé l'année aux trois valeurs 1999, 2009, puis 2019.
		\begin{center}\begin{tabular}{cn{2}{1}}
			\toprule
			Année & {Moyenne} \\
			1999 &    25.831762 \\
			2009 &    29.661775 \\
			2019 &    31.159518 \\
			\bottomrule
		\end{tabular}\end{center}
	\item La moyenne est en défaveur des femmes, et augmente dans le temps, donc oui.
		\end{enumerate}
		\stepcounter{enumi}
	\item La plus égalitaire est Téva (\numprint{49.507175}\%), la moins égalitaire est Canal+ Sport (\numprint{5.136651}\%).

		Voici le programme utilisé pour répondre.

		\begin{lstlisting}
		import pandas
		parole = pandas.read_csv("20190308-years.csv")
		recherche = parole.loc[
			(parole['year'] == 2019),
			["channel_name", "women_expression_rate"],
		]
		tri = recherche.sort_values(
			by="women_expression_rate",
			)
		print(tri)
		\end{lstlisting}
\end{enumerate}


\end{document}
