%%%%%%%%%%%%%%%%%%%%%%%%%%%%%%%%%%%%%%%%%%%%%%%%%%%%%%%%%%%%%%%%%%%%%%%%%%%%%%%%
% Copyright 2020 Louis Paternault --- http://snt.ababsurdo.fr
%
% Publié sous licence Creative Commons Attribution-ShareAlike 4.0 International (CC BY-SA 4.0)
% http://creativecommons.org/licenses/by-sa/4.0/deed.fr
%%%%%%%%%%%%%%%%%%%%%%%%%%%%%%%%%%%%%%%%%%%%%%%%%%%%%%%%%%%%%%%%%%%%%%%%%%%%%%%%

% Pour compiler :
%$ lualatex $basename

\documentclass[12pt]{article}

\usepackage{calc}
\usepackage[
  paperwidth={21cm/2},
  paperheight={29.7cm/4},
  margin=6mm,
  includehead,
]{geometry}
\usepackage{evaluation}

\title{Photographie --- Algo}
\hypersetup{
  unicode=true,
  urlcolor=cyan,
  pdfauthor={Louis Paternault},
  pdfproducer={© Louis Paternault — CC-BY-SA-4.0 — http://snt.ababsurdo.fr},
  hidelinks,
}

\newcommand{\points}{\parbox[t]{13cm}{\dotfill}}

\begin{document}

\vspace*{\stretch{1}}

\rowcolors{1}{}{lightgray}
\begin{tabularx}{\textwidth}{Xc}
  \toprule
  Python : Assombrir & …/1 \\
  Python : Éclaircir & …/1 \\
  Python : Symétrie & …/1 \\
  Python : Niveaux de gris & …/1 \\
  Python : Inverser les couleurs & …/1 \\
  Python : Pivoter & …/1 \\
  \midrule
  \textbf{Total} & \textbf{…/6} \\

  \bottomrule
\end{tabularx}

\end{document}
