\input{preambule_damien2.tex}   

%%%%%%%%%%%%%%%%%%%%%%%%%%%%%%%%%%%%%%%%%%%
%****************TABLEAUX*******************************%
%%%%%%%%%%%%%%%%%%%%%%%%%%%%%%%%%%%%%%%%%%%

%%%%%%%%%%%%%%%% TABLEAU SIGNE ET VARIATION %%%%%%%%%%%%%%
%\begin{center}
%%\vspace{0.3cm}
%\begin{tikzpicture}
%\tkzTabInit[lgt=2,espcl=2]{$x$ /.8 , $d(x)$ /1.5}{$0$ , $4$,$16$,$24$}
%\tkzTabLine[]{,+,z,-,}
%\tkzTabVar[]{+/ $8$,-/$6$,+/$19$,-/$8$}
%%\tkzTabVal[draw]{1}{2}{0.5}{0}{0}
%\tkzTabIma{4}{6}{5}{$\frac{16}{3}$}
%\end{tikzpicture}
%\end{center}

%%%%%%%%%%%%%%% TABLEAU %%%%%%%%%%%%%%%%%%%%%%%%%%%%%%
%\renewcommand{\arraystretch}{hauteur}
%\begin{center}
%\begin{tabularx}{0.94\linewidth}{|c|*{7}{>{\centering \arraybackslash}X|}}
%\hline 
%
%\hline 
%\end{tabularx} 
%\end{center}

%%%%%%%%%%%%%%%%%%%%%%%%%%%%%%%%%%%%%%%%%%%
%****************** GRAPHIQUE FCT**********************%
%%%%%%%%%%%%%%%%%%%%%%%%%%%%%%%%%%%%%%%%%%%

%\psset{xunit=0.75cm,yunit=0.75cm,algebraic=true,dimen=middle,dotstyle=o,dotsize=5pt 0,linewidth=1.4pt,arrowsize=2pt 2,arrowinset=0.25}
%\def\xmin{-9} \def\xmax{9} \def\ymin{-10.5} \def\ymax{6.5} \def\dx{0.5} \def\dy{0.5}
%\begin{center}
%		\begin{pspicture*}(\xmin,\ymin)(\xmax,\ymax)
%		\multips(0,\ymin)(0,\dy){35}{\psline[linestyle=dashed,linecap=1,dash=1.5pt 1.5pt,linewidth=0.4pt,linecolor=gray]{c-c}(\xmin,0)(\xmax,0)}
%		\multips(\xmin,0)(\dx,0){38}{\psline[linestyle=dashed,linecap=1,dash=1.5pt 1.5pt,linewidth=0.4pt,linecolor=lightgray]{c-c}(0,\ymin)(0,\ymax)}
%		\psaxes[labels=all,labelFontSize=\scriptstyle,labelsep=2pt,xAxis=true,yAxis=true,Dx=1,Dy=1,ticksize=-2pt 0,subticks=2,showorigin=False]{->}(0,0)(\xmin,\ymin)(\xmax,\ymax)[$x$,0][$y$,90]
%\psclip{%
%\psframe[linestyle=none](\xmin,\ymin)(\xmax,\ymax)}
%		\uput[dl](0,0){\scriptsize $0$}
%		\psplot[linewidth=1.4pt,plotpoints=200, linecolor=blue]{\xmin}{\xmax}{2.79^x}
%\endpsclip
%		\end{pspicture*}
%\end{center}

%%%%%%%%%%%%%%%%%%%%%%%%%%%%%%%%%%%%%%%%%%%
% *******************MINTED *****************************%
%%%%%%%%%%%%%%%%%%%%%%%%%%%%%%%%%%%%%%%%%%%
\usepackage{minted}

\renewcommand{\theFancyVerbLine}{\textcolor{gray}{\tiny \oldstylenums{\arabic{FancyVerbLine}}}}

\definecolor{bg}{rgb}{0.95,0.95,0.95}

\newminted[python]{python}{fontfamily=tt,linenos=true,autogobble,mathescape=true,python3,fontsize=\small, tabsize=4, samepage=true, rulecolor=gray, numbersep=2pt, bgcolor=bg}

\newminted[pythonsansnum]{python}{fontfamily=tt,linenos=false,autogobble,mathescape=true,python3,fontsize=\small, tabsize=4, samepage=true, rulecolor=gray, numbersep=2pt, bgcolor=bg}

\newmintinline{python}{fontfamily=tt,linenos=true,autogobble,mathescape=true,python3}


\newmintedfile[pythonexternal]{python}{fontfamily=tt,linenos=false,autogobble,mathescape=true,python3}

\newminted[html]{html}{fontfamily=courier, fontsize=\footnotesize, rulecolor=gray, framerule=1.5pt, mathescape=true, texcomments=true, autogobble, tabsize=4, numbersep=8pt}

\newmintinline{html}{fontfamily=courier, fontsize=\small}

\newminted[css]{css}{fontfamily=courier, fontsize=\footnotesize, rulecolor=gray, framerule=1.5pt, mathescape=true, texcomments=true, autogobble, tabsize=4, numbersep=8pt}

\newmintinline{css}{fontfamily=courier, fontsize=\small}

\newminted[algo]{bbcode}{fontfamily=courier, fontsize=\footnotesize, rulecolor=gray, framerule=1.5pt, mathescape=true, texcomments=true, autogobble, linenos=true, tabsize=4, numbersep=8pt}

%%%%%%%%%%%%%%%%%%%%%%%%%%%%%%%%%%%%%%%%%%%
% *******************EnT�tes et Pieds de page *****************************%
%%%%%%%%%%%%%%%%%%%%%%%%%%%%%%%%%%%%%%%%%%%
\pagestyle{fancy}
\setlength{\headheight}{40pt} % Haut de page
\renewcommand{\headrulewidth}{0.8pt}
\setlength{\textheight}{26cm}
\lhead{\footnotesize \em Nom : }
%\chead{\thepage/\pageref{LastPage}}
\rhead{}
\renewcommand{\footrulewidth}{1pt}
\lfoot{2GT --- SNT --- �valuation T�che finale Web}
\cfoot{}
%\rfoot{\thepage/\pageref{LastPage}}

\begin{document}

\section*{�valuation T�che finale SNT --- Web}

\subsection*{Obligatoire}

\renewcommand{\arraystretch}{1.5}
\begin{center}
\begin{tabularx}{1\linewidth}{l*{2}{>{\centering \arraybackslash}X}}
\hline
Le titre de la page (qui s'affiche sur l'onglet) est correct & \dots / 1 & \dots / 1 \\
Le \texttt{charset} choisi est bien \texttt{utf-8} & \dots / 1 & \dots / 1 \\
La page contient un titre \texttt{<h1>} et plusieurs titres \texttt{<h2>} corrects & \dots / 1 & \dots / 1 \\
La page contient quelques paragraphes r�dig�s utilisant correctement la balise \texttt{<p>} & \dots / 1 & \dots / 1 \\
La page contient une image & \dots / 1 & \dots / 1 \\
La page contient un lien vers une autre page & \dots / 1 & \dots / 1 \\
La page contient une liste & \dots / 1 & \dots / 1 \\
Le lien vers le fichier CSS est correct & \dots / 1 & \dots / 1 \\
Le travail a �t� rendu � temps et le nom appara�t dans le code source & \dots / 1 & \dots / 1 \\
Le travail a �t� rendu sous la forme demand�e & \dots / 1 & \dots / 1 \\
&& \\
\textbf{\Large Sous-total} & \dots \textbf{\Large / 10} & \dots \textbf{\Large / 10} \\
\end{tabularx} 
\end{center}

\subsection*{Optionnels}
\renewcommand{\arraystretch}{1.5}
\begin{center}
\begin{tabularx}{1\linewidth}{l*{2}{>{\centering \arraybackslash}X}}
\hline
\multicolumn{1}{l}{\textit{R�alisez quelques objectifs parmi les suivants }} \\
\multicolumn{1}{l}{\textit{(qui sont � peu pr�s class�s par ordre de difficult� croissante).}} \\ \hline
Au moins deux balises utilis�es parmi : \texttt{<em>, <i>, <b>, <strong>} & \dots / 1 & \dots / 1 \\
L'auteur (ou auteure) de l'image est nomm�.e ou l'image est libre & \dots / 1 & \dots / 1 \\
Le fichier \texttt{CSS} a �t� modifi� pour changer l'apparence de la page Web & \dots / 2 & \dots / 2 \\
L'esth�tique de la page Web est \og professionelle \fg{} & \dots / 1 & \dots / 1 \\
\textbf{\Large Sous-total} & \dots \textbf{\Large / 5} & \dots \textbf{\Large / 5} \\\hline
&& \\
\textbf{\LARGE Total} &  \dots \textbf{\LARGE / 15} & \dots \textbf{\LARGE / 15} \\
\end{tabularx} 
\end{center}

\end{document}