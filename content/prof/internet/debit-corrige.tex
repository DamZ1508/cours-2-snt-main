%%%%%%%%%%%%%%%%%%%%%%%%%%%%%%%%%%%%%%%%%%%%%%%%%%%%%%%%%%%%%%%%%%%%%%%%%%%%%%%%
% Copyright 2022 Louis Paternault --- http://snt.ababsurdo.fr
%
% Publié sous licence Creative Commons Attribution-ShareAlike 4.0 International (CC BY-SA 4.0)
% http://creativecommons.org/licenses/by-sa/4.0/deed.fr
%%%%%%%%%%%%%%%%%%%%%%%%%%%%%%%%%%%%%%%%%%%%%%%%%%%%%%%%%%%%%%%%%%%%%%%%%%%%%%%%

% Pour compiler :
%$ lualatex $basename

\documentclass[12pt]{article}

\usepackage{2122-pablo}
\usepackage{2122-pablo-paternault}
\usepackage{2122-pablo-math}

\usepackage[
  a4paper,
  margin=10mm,
  %headsep=5mm,
  includehead,
]{geometry}
\usepackage{2122-pablo-header}
\fancyhead[L]{\textsc{Snt --- Internet}}
\fancyhead[R]{\textsc{5 --- Réseau physique : Débit}}

\newcommand{\red}[1]{\textcolor{red}{#1}}

\begin{document}

\begin{enumerate}
  \item
    \begin{enumerate}
      \item \emph{Convertir en octets : \SI{1}{ko}, \SI{1}{Mo}, \SI{1}{Go}, \SI{1}{To}.}

        \begin{itemize}
          \item \SI{1}{ko} = \enquote{1 kilooctets} = \numprint{1000} octets ;
          \item \SI{1}{Mo} = \enquote{1 mégaoctets} = \numprint{1000} kilooctets = \numprint{1000000} (un million) octets ;
          \item \SI{1}{Go} = \enquote{1 gigaoctets} = \numprint{1000} mégaoctets = \numprint{1000000000} (un milliard) octets ;
          \item \SI{1}{To} = \enquote{1 téraoctets} = \numprint{1000} gigaoctets = \numprint{1000000000} (mille milliards) octets.
        \end{itemize}
      \item \emph{Convertir en gigaoctets : \SI{2345,6}{ko}.} On utilise un tableau de conversion d'unités (que vous avez découvert en école primaire, mais qui reste un excellent outil).

        \begin{center}\begin{tabular}{*{13}{|p{6mm}}}
            To &&& Go &&& Mo &&& ko &&& o \\
            &&&  \red{0,} &\red{0}&\red{0}&   2&3&4&5,&6&&
        \end{tabular}\end{center}
        Donc $\SI{2345,6}{ko}=\SI{0,0023456}{Go}$.
    \end{enumerate}
  \item Commençons par remarquer que si : $\text{débit}=\frac{\text{taille}}{\text{temps}}$, alors :
    \[\text{temps}=\frac{\text{taille}}{\text{débit}} \text{ et } \text{taille}=\text{débit}\times\text{temps}\]
    Il faudra vérifier que les unités sont cohérentes entre elles.
    \begin{enumerate}
      \item \emph{Pendant une vidéo \emph{live} de deux minutes avec un débit de \SI{1,5}{Mo/s}, quelle quantité de données a été transmise ?}

        On connait la durée du téléchargement (deux minutes, soit 120 secondes), ainsi que la vitesse (\SI{1,5}{Mo/s}). La quantité de données transmises est donc $120\times1,5=\SI{180}{Mo}$.
      \item \emph{Dans les années 90, la plupart des foyers se connectaient à Internet avec des \enquote{modems \SI{56}{k}}, qui avaient un débit théorique maximum de \SI{56}{ko/s}. Combien de temps fallait-il pour télécharger une photo de \SI{2}{Mo} ?}

        On connait la quantité de données (\SI{2}{Mo}, soit \SI{2000}{ko}), ainsi que le débit (\SI{56}{ko/s}). La durée de téléchargement est donc :
        $\frac{2000}{56}\approx35,7$. Il fallait donc \numprint{35,7} secondes environ pour télécharger cette image.
    \end{enumerate}
\end{enumerate}

\end{document}
