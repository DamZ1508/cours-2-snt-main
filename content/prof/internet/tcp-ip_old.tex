%%%%%%%%%%%%%%%%%%%%%%%%%%%%%%%%%%%%%%%%%%%%%%%%%%%%%%%%%%%%%%%%%%%%%%%%%%%%%%%%
% Copyright 2022 Louis Paternault --- http://snt.ababsurdo.fr
%
% Publié sous licence Creative Commons Attribution-ShareAlike 4.0 International (CC BY-SA 4.0)
% http://creativecommons.org/licenses/by-sa/4.0/deed.fr
%%%%%%%%%%%%%%%%%%%%%%%%%%%%%%%%%%%%%%%%%%%%%%%%%%%%%%%%%%%%%%%%%%%%%%%%%%%%%%%%

% Pour compiler :
%$ lualatex $basename

\documentclass[11pt]{article}

\usepackage{2122-pablo}
\usepackage{2122-pablo-paternault}

\usepackage[
  a6paper,
  landscape,
  margin=5mm,
  headsep=5mm,
  includehead,
]{geometry}
\usepackage{2122-pablo-header}
\fancyhead[L]{\textsc{Snt --- Internet}}
\fancyhead[R]{\textsc{3 --- Modèle TCP/IP}}

\begin{document}

Les données qui transitent sur Internet sont découpées en petits morceaux, transmis séparément.

\paragraph{Préparation (protocole TCP) :} Les données sont découpées en \emph{paquets}, qui contiennent : les données, et le numéro du paquet.

\paragraph{Transmission (protocole IP) :} Ces paquets sont envoyés au destinataire, séparément : les différents paquets peuvent emprunter des chemins différents, arriver dans le désordre, être perdus et ré-envoyés, etc.

Le destinataire est identifié par son \emph{adresse IP}, de la forme :
\begin{itemize}
  \item pour l'IPv4 : \texttt{185.75.143.24} (quatre nombres entre 0 et 255) ;
  \item pour l'IPv6 : \texttt{2a02:ec80:600:ed1a::1} (huit groupes de 16 bits, représentés en hexadécimal).
  \end{itemize}

\paragraph{Réception (protocole TCP) :} Les paquets sont reçus, remis dans le bon ordre (grâce aux numéros contenus dans les paquets), et assemblés pour reconstituer les données d'origine.

\end{document}
