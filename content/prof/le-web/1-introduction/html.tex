\input{preambule_damien2.tex}     

%%%%%%%%%%%%%%%%%%%%%%%%%%%%%%%%%%%%%%%%%%%
%****************TABLEAUX*******************************%
%%%%%%%%%%%%%%%%%%%%%%%%%%%%%%%%%%%%%%%%%%%

%%%%%%%%%%%%%%%% TABLEAU SIGNE ET VARIATION %%%%%%%%%%%%%%
%\begin{center}
%%\vspace{0.3cm}
%\begin{tikzpicture}
%\tkzTabInit[espcl=2]{$x$ /.8 , $d(x)$ /1.5}{$0$ , $4$,$16$,$24$}
%\tkzTabLine[]{,+,z,-,}
%\tkzTabVar[]{+/ $8$,-/$6$,+/$19$,-/$8$}
%%\tkzTabVal[draw]{1}{2}{0.5}{0}{0}
%\end{tikzpicture}
%\end{center}

%%%%%%%%%%%%%%% TABLEAU %%%%%%%%%%%%%%%%%%%%%%%%%%%%%%
%\renewcommand{\arraystretch}{hauteur}
%\begin{center}
%\begin{tabularx}{0.94\linewidth}{|c|*{7}{>{\centering \arraybackslash}X|}}
%\hline 
%
%\hline 
%\end{tabularx} 
%\end{center}

%%%%%%%%%%%%%%%%%%%%%%%%%%%%%%%%%%%%%%%%%%%
%****************** GRAPHIQUE FCT**********************%
%%%%%%%%%%%%%%%%%%%%%%%%%%%%%%%%%%%%%%%%%%%

%\psset{xunit=0.75cm,yunit=0.75cm,algebraic=true,dimen=middle,dotstyle=o,dotsize=5pt 0,linewidth=1.4pt,arrowsize=2pt 2,arrowinset=0.25}
%\def\xmin{-9} \def\xmax{9} \def\ymin{-10.5} \def\ymax{6.5} \def\dx{0.5} \def\dy{0.5}
%\begin{center}
%		\begin{pspicture*}(\xmin,\ymin)(\xmax,\ymax)
%		\multips(0,\ymin)(0,\dy){35}{\psline[linestyle=dashed,linecap=1,dash=1.5pt 1.5pt,linewidth=0.4pt,linecolor=lightgray]{c-c}(\xmin,0)(\xmax,0)}
%		\multips(\xmin,0)(\dx,0){38}{\psline[linestyle=dashed,linecap=1,dash=1.5pt 1.5pt,linewidth=0.4pt,linecolor=lightgray]{c-c}(0,\ymin)(0,\ymax)}
%		\psaxes[labels=all,labelFontSize=\scriptstyle,labelsep=2pt,xAxis=true,yAxis=true,Dx=1,Dy=1,ticksize=-2pt 0,subticks=2,showorigin=False]{->}(0,0)(\xmin,\ymin)(\xmax,\ymax)[{$x$,0][$y$,90]
%\psclip{%
%\psframe[linestyle=none](\xmin,\ymin)(\xmax,\ymax)}
%		\uput[dl](0,0){\scriptsize $0$}
%		\psplot[linewidth=1.4pt,plotpoints=200, linecolor=blue]{\xmin}{\xmax}{2.79^x}
%}\endpsclip
%		\end{pspicture*}
%\end{center}

%%%%%%%%%%%%%%%%%%%%%%%%%%%%%%%%%%%%%%%%%%%
% *******************MINTED *****************************%
%%%%%%%%%%%%%%%%%%%%%%%%%%%%%%%%%%%%%%%%%%%
\usepackage{minted}

\renewcommand{\theFancyVerbLine}{\textcolor{gray}{\tiny \oldstylenums{\arabic{FancyVerbLine}}}}

\definecolor{bg}{rgb}{0.95,0.95,0.95}

\newminted[python]{python}{fontfamily=tt,linenos=true,autogobble,mathescape=true,python3,fontsize=\small, tabsize=4, samepage=true, rulecolor=gray, numbersep=2pt, bgcolor=bg}

\newmintinline{python}{fontfamily=tt,linenos=true,autogobble,mathescape=true,python3}

\newmintedfile[pythonexternal]{python}{fontfamily=tt,linenos=true,autogobble,mathescape=true,python3}

\newminted[html]{html}{fontfamily=courier, fontsize=\footnotesize, rulecolor=gray, framerule=1.5pt, mathescape=true, texcomments=true, autogobble, tabsize=4, numbersep=8pt}

\newmintinline{html}{fontfamily=courier, fontsize=\small}

\newminted[css]{css}{fontfamily=courier, fontsize=\footnotesize, rulecolor=gray, framerule=1.5pt, mathescape=true, texcomments=true, autogobble, tabsize=4, numbersep=8pt}

\newmintinline{css}{fontfamily=courier, fontsize=\small}

\newminted[algo]{bbcode}{fontfamily=courier, fontsize=\footnotesize, rulecolor=gray, framerule=1.5pt, mathescape=true, texcomments=true, autogobble, linenos=true, tabsize=4, numbersep=8pt}

%%%%%%%%%%%%%%%%%%%%%%%%%%%%%%%%%%%%%%%%%%%
% *******************EnT�tes et Pieds de page *****************************%
%%%%%%%%%%%%%%%%%%%%%%%%%%%%%%%%%%%%%%%%%%%
\pagestyle{fancy}
\setlength{\headheight}{40pt} % Haut de page
\renewcommand{\headrulewidth}{0.0pt}
\setlength{\textheight}{26cm}
\lhead{}
%\chead{\thepage/\pageref{LastPage}}
\rhead{}
\renewcommand{\footrulewidth}{1pt}
\lfoot{2GT --- SNT}
\cfoot{}
%\rfoot{\thepage/\pageref{LastPage}}


%%%%%%%%%%%%%%%%%%%%%%%%%%%%%%%%%%%%%%%%%%%%%%



\usepackage{xspace}
\newcommand{\ellipse}{[...]\xspace}

% mdframed
\usepackage{mdframed}
\mdfdefinestyle{citationstyle}{%
  outerlinewidth=.5em,
  outerlinecolor=white,%
  leftmargin=-1em,
  rightmargin=-1em,%
  middlelinewidth=1.2pt,
  roundcorner=5pt,
  linecolor=black,
  skipbelow=10cm,
  innerbottommargin=5pt,
  innerleftmargin=5pt,
  innerrightmargin=5pt,
  innertopmargin=5pt,
}
%%%%%%%

\begin{document}

\begin{titre}{2NDE}{Seconde - SNT}
\centering
{\Large \textsc{\textbf{Th�me : Le web --- Les Balises HTML}}}
\end{titre}

%\lstset{language=html, basicstyle=\ttfamily}

%\usepackage{amssymb}
\newcommand{\valide}{
  $\square$ Valide
  \hspace{1cm}
  $\square$ Erreur : \dotfill
}

\bigskip

\exo 
Pour chacun de ces extraits de code HTML, dire s'il est valide, ou s'il y a une erreur, en pr�cisant laquelle.

\begin{enumerate}
  \item \begin{html}
<p>Bonjour, <em>monde</em> !</p>
\end{html}
\valide

\medskip

  \item \begin{html}
    <em>Titi <u>toto</u> tata</u></em>
\end{html}
\valide

\medskip

  \item \begin{html}
    <ul>
    <li>Bonjour</li>
    <li><em>au</em> revoir</li>
\end{html}
\valide

\medskip

  \item \begin{html}
    <p>Tagada <em>tsoin</em>
    <u>tsoin</u><p>
\end{html}
\valide

\medskip

  \item \begin{html}
    <div><a>chien <u>chat</u> cheval
    <em>lapin</em> </div> </a>
\end{html}
\valide

\medskip

  \item \begin{html}
    <h1>
    pif <em>paf <b>pouf</b></em> plop
    </h1>
\end{html}
\valide

\end{enumerate}

\exo
Le code HTML ci-dessous comprend 7 erreurs de syntaxe. Trouvez-les et corrigez-les. 

\begin{html}
<!DOCTYPE html>
<html lang="fr">
     <head>
          <meta charset="utf-8">
          <titre>Ma super page</titre>
     </head>
     <body>
          <header>	<!--Bandeau du haut-->
               <p>This is mon blog</p>
               <img src="./logo.jpg">
          </header>
          <nav>		<!--Liens sur le c�t�-->
          	  <ul>
               <li><a href="http://www.unsite.com">Site d'un copain</a></li>
               <li><a href="http://www.autresite.fr">Site d'une copine</li></a>
               </ul>
          </nav>
          <section>	<!--Texte au centre-->
               <article>
                    <h1>D�j� la rentr�e</h1>
                    <p>Aujourd'hui c'est la rentr�e. C'est nul, il pleut. Et le 
                    pire c'est que j'ai M. Ess�netais en prof... trop la loose. </article></p>

               <article>
                    <h1>C'est l'�t�</h1>
                    <p>Hier j'ai mang� une glace allong� sur le sable. 
                    Il faisait chaud. J'ai pris un <strong>coup de soleil. -_- </p>
               </article>
          <section>

</html>
\end{html}

\end{document}
