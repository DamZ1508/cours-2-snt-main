%%%%%%%%%%%%%%%%%%%%%%%%%%%%%%%%%%%%%%%%%%%%%%%%%%%%%%%%%%%%%%%%%%%%%%%%%%%%%%%%
% Copyright 2020 Louis Paternault --- http://ababsurdo.fr
%
% Publi� sous licence Creative Commons Attribution-ShareAlike 4.0 International (CC BY-SA 4.0)
% http://creativecommons.org/licenses/by-sa/4.0/deed.fr
%%%%%%%%%%%%%%%%%%%%%%%%%%%%%%%%%%%%%%%%%%%%%%%%%%%%%%%%%%%%%%%%%%%%%%%%%%%%%%%%

% Pour compiler :
%$ lualatex $basename

\documentclass[14pt]{beamer}
\usetheme{PaloAlto}
\usepackage[latin9]{inputenc}
\usepackage[french]{babel}
\usepackage[T1]{fontenc}
\usepackage{amsmath}
\usepackage{amsfonts}
\usepackage{amssymb}
\usepackage{graphicx}
\usepackage{multicol}
\usefonttheme[onlymath]{serif}
\usepackage{pstricks,pst-plot,pst-text,pst-tree,pst-eps,pst-fill,pst-eucl,pst-node,pst-math,pst-blur}
\input{tabvar.tex}
\usepackage{pstricks-add}
\usepackage{pst-3dplot}				% schma 3D
\usepackage{pst-func} % rep\'esentation loi binomiale, normale
\usepackage{wrapfig}
\usepackage{listings}
\usepackage[autolanguage,np]{numprint}
\usepackage{cancel}    

%%%%%%%%%%%%%%%%%%%%%%%%%%%%%%%%%%%%%%%%%%%
% *******************MINTED *****************************%
%%%%%%%%%%%%%%%%%%%%%%%%%%%%%%%%%%%%%%%%%%%
\usepackage{minted}
%
\renewcommand{\theFancyVerbLine}{\textcolor{gray}{\tiny \oldstylenums{\arabic{FancyVerbLine}}}}
%
\definecolor{bg}{rgb}{0.95,0.95,0.95}
%
\newminted[python]{python}{fontfamily=tt,linenos=true,autogobble,mathescape=true,python3,fontsize=\small, tabsize=4, samepage=true, rulecolor=gray, numbersep=2pt, bgcolor=bg}
%
\newmintinline{python}{fontfamily=tt,linenos=true,autogobble,mathescape=true,python3}
%
\newmintedfile[pythonexternal]{python}{fontfamily=tt,linenos=true,autogobble,mathescape=true,python3}

\newminted[html]{html}{fontfamily=courier, fontsize=\footnotesize, rulecolor=gray, framerule=1.5pt, mathescape=true, texcomments=true, autogobble, tabsize=4, numbersep=8pt}

\newmintinline{html}{fontfamily=courier, fontsize=\small}

\newminted[css]{css}{fontfamily=courier, fontsize=\footnotesize, rulecolor=gray, framerule=1.5pt, mathescape=true, texcomments=true, autogobble, tabsize=4, numbersep=8pt}

\newmintinline{css}{fontfamily=courier, fontsize=\small}

\newminted[algo]{bbcode}{fontfamily=courier, fontsize=\footnotesize, rulecolor=gray, framerule=1.5pt, mathescape=true, texcomments=true, autogobble, linenos=true, tabsize=4, numbersep=8pt}

%\usepackage{2021-pablo}
%\usepackage{2021-pablo-beamer}
%\usepackage{2021-pablo-paternault}
%\usepackage{2021-pablo-listings}

\title{Introduction au HTML}
%\setbeamertemplate{navigation symbols}{} 
\institute{\textbf{\Large 2GT --- SNT}} 
\subtitle{Les Balises HTML} 
\date{}

\begin{document}
\begin{frame}[fragile]{Structure g�n�rale d'une page web}
\begin{html}
<!DOCTYPE html>
<html lang="fr">
  <head>
    <meta charset="utf-8">
    <title>Titre</title>
  </head>
  <body>
    Bonjour !
  </body>
</html>
\end{html}
\end{frame}

\begin{frame}[fragile]{Explication des balises}
  Des mots \underline{soulign�s}, \textbf{gras} ou en \emph{italique}.

  \vfill

\begin{html}
<p>Des mots
<u>soulign�s</u>,
<b>gras</b>,
ou en <em>italique</em>.
</p>
\end{html}
\end{frame}
%
\begin{frame}[fragile]{Exemple avec la pile}
\begin{minipage}{0.6\textwidth}
\begin{html}
<p>
Tous les <u>�tres humains</u>
naissent <em>libres
et <b>�gaux</b></em>
<b>en dignit� et en
<u>droits</u></b>.
</p>
\end{html}
\end{minipage}
\hfill
\begin{minipage}{0.3\textwidth}
\psset{unit=0.35cm}
 \begin{pspicture}(-2,-5)(2,15)
  \psline(-1,12)(-1,-4)(1,-4)(1,12)
    \end{pspicture}
\end{minipage}
\end{frame}

\begin{frame}[fragile]{Exemple d'erreurs - 1}
\begin{minipage}{0.6\textwidth}
\begin{html}
<p>
Ils sont <b>dou�s de
<em>raison</b>
et de conscience</em>.
</p>
\end{html}
\end{minipage}
\hfill
\begin{minipage}{0.3\textwidth}
\psset{unit=0.35cm}
 \begin{pspicture}(-2,-5)(2,15)
  \psline(-1,12)(-1,-4)(1,-4)(1,12)
    \end{pspicture}
\end{minipage}
\end{frame}

\begin{frame}[fragile]{Exemple d'erreurs - 2}
\begin{minipage}{0.6\textwidth}
\begin{html}
<b>Tout</b> individu
a <em>droit � la <u>vie</u>.
\end{html}
\end{minipage}
\hfill
\begin{minipage}{0.3\textwidth}
\psset{unit=0.35cm}
 \begin{pspicture}(-2,-5)(2,15)
  \psline(-1,12)(-1,-4)(1,-4)(1,12)
    \end{pspicture}
\end{minipage}
\end{frame}

\end{document}
