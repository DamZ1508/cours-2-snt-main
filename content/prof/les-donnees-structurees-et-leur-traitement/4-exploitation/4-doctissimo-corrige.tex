%%%%%%%%%%%%%%%%%%%%%%%%%%%%%%%%%%%%%%%%%%%%%%%%%%%%%%%%%%%%%%%%%%%%%%%%%%%%%%%%
% Copyright 2019 Louis Paternault --- http://snt.ababsurdo.fr
%
% Publié sous licence Creative Commons Attribution-ShareAlike 4.0 International (CC BY-SA 4.0)
% http://creativecommons.org/licenses/by-sa/4.0/deed.fr
%%%%%%%%%%%%%%%%%%%%%%%%%%%%%%%%%%%%%%%%%%%%%%%%%%%%%%%%%%%%%%%%%%%%%%%%%%%%%%%%

% Pour compiler :
%$ lualatex $basename

\documentclass[12pt]{article}

\usepackage[a4paper]{geometry}
\usepackage{textcomp}
\usepackage[francais]{babel}
\usepackage{csquotes}
\usepackage{hyperref}
\hypersetup{
  unicode=true,
  urlcolor=cyan,
  pdfauthor={Louis Paternault},
  pdfproducer={© Louis Paternault — CC-BY-SA-4.0 — http://snt.ababsurdo.fr},
  hidelinks,
}

\pagestyle{empty}

\begin{document}

\section*{Étude de documents --- Exploitation des données personnelles --- Corrigé}

Les données personnelles des utilisateurs peuvent-être rentables pour l'éditeur d'un site web.
\emph{En effet,} les sites web dont l'accès est gratuit (comme Doctissimo) génèrent des revenus en affichant de la publicité à leurs visiteurs. Les données personnelles récoltées par l'éditeur du site web permettent à celui-ci de vendre aux annonceurs des profils d'utilisateurs plus précis, pour que les publicités soient plus efficaces.
\emph{Ainsi,} les visiteurs \enquote{fournissent au site des informations nombreuses sur eux-mêmes}. Et ces informations \enquote{dessinent des profils toujours plus précis, et donc toujours plus rentables}. Un des directeurs du groupe TF1 admet que \enquote{[leur] objectif est de valoriser au mieux [les données des utilisateurs]}.
\emph{Donc} les données personnelles des utilisateurs peuvent-être rentables pour l'éditeur d'un site web.

\end{document}
