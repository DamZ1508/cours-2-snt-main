%%%%%%%%%%%%%%%%%%%%%%%%%%%%%%%%%%%%%%%%%%%%%%%%%%%%%%%%%%%%%%%%%%%%%%%%%%%%%%%%
% Copyright 2019-2021 Louis Paternault --- http://ababsurdo.fr
%
% Publié sous licence Creative Commons Attribution-ShareAlike 4.0 International (CC BY-SA 4.0)
% http://creativecommons.org/licenses/by-sa/4.0/deed.fr
%%%%%%%%%%%%%%%%%%%%%%%%%%%%%%%%%%%%%%%%%%%%%%%%%%%%%%%%%%%%%%%%%%%%%%%%%%%%%%%%

% Pour compiler :
%$ lualatex $basename

\documentclass[11pt]{article}

\usepackage[
  a4paper,
  margin=.6cm,
  includehead,
]{geometry}
\setlength{\headsep}{10pt}
\usepackage{fancyhdr}
\fancyfoot{}
\fancyhead[L]{\textsc{SNT > Données}}
\fancyhead[C]{\textbf{Exploitation des données personnelles}}
\fancyhead[R]{\textsc{Étude de documents}}
\pagestyle{fancy}

\usepackage{textcomp}
\usepackage[francais]{babel}
\usepackage{csquotes}
\usepackage[shortlabels]{enumitem}
\usepackage{hyperref}
\hypersetup{
  unicode=true,
  urlcolor=cyan,
  pdfauthor={Louis Paternault},
  pdfproducer={© Louis Paternault — CC-BY-SA-4.0 — http://snt.ababsurdo.fr},
  hidelinks,
}

\usepackage{fontawesome}
\newcommand{\cuthere}{\noindent%
  \dotfill \raisebox{-.25em}{\faScissors}%
  \dotfill \raisebox{-.25em}{\faScissors}%
  \dotfill \raisebox{-.25em}{\faScissors}%
  \dotfill \raisebox{-.25em}{\faScissors}%
  \dotfill}

\usepackage{xspace}
\newcommand{\ellipse}{[…]\xspace}

%%%%%%%
% mdframed
\usepackage[framemethod=TikZ]{mdframed}
\mdfdefinestyle{citationstyle}{%
  outerlinewidth=.5em,
  outerlinecolor=white,%
  leftmargin=-1em,
  rightmargin=-1em,%
  middlelinewidth=1.2pt,
  roundcorner=5pt,
  linecolor=black,
  skipbelow=10cm,
  innerbottommargin=5pt,
  innerleftmargin=5pt,
  innerrightmargin=5pt,
  innertopmargin=5pt,
}
%%%%%%%

\begin{document}

\noindent Lire le document, puis répondre à la question suivante sous la forme d'un paragraphe argumenté.

\begin{mdframed}[style=citationstyle]
  \textbf{Question :} Les données personnelles des utilisateurs peuvent-elles être rentables pour l'éditeur d'un site web ?
\end{mdframed}

\begin{em}
\noindent \emph{Contexte :} Créé en 2000, Doctissimo est un site web francophone consacré à la santé et au bien-être. Il a été racheté en 2018 par le groupe TF1.
\end{em}

\begin{mdframed}[style=citationstyle]
  \textbf{Sophie Eustache. \emph{La \enquote{patiente informée}, une bonne affaire}. Le Monde diplomatique, mai 2019, \no{782}, page 23.}\hfill\url{https://www.monde-diplomatique.fr/2019/05/EUSTACHE/59878}

  ~\hrule~

%Un sol couvert de (fausses ?) peaux de vache ainsi que des fauteuils douillets et luxueux accueillent les visiteurs du groupe de presse Aufeminin (Aufeminin, Marmiton, My Little Paris, etc.), propriété de TF1. Tout semble recouvert d'une couche de vernis. C'est dans ces locaux situés rue Saint-Fiacre, à Paris, qu'a emménagé fin 2018 l'équipe du site Doctissimo, racheté par TF1 pour 15 millions d'euros. \emph{\enquote{Nos hôtes ont privilégié l'esthétique au détriment du pratique}}, plaisante David Bême, rédacteur en chef de Doctissimo, en s'installant dans l'un des fauteuils de jardin qui entourent une table de réunion.

%Créé en 2000,
%%par MM. Laurent Alexandre et Claude Malhuret, deux médecins proches de la droite libérale,
%\ellipse
%Doctissimo est l'un des pionniers français de l'information en ligne sur la santé destinée au grand public. Les fondateurs ambitionnaient de \emph{\enquote{démocratiser l'accès à la santé}}, mais aussi, explique l'ancienne présidente du groupe, Mme Valérie Brouchoud, de \emph{\enquote{rendre le patient responsable, en considérant qu'un patient éclairé peut avoir un dialogue plus constructif avec son médecin, une meilleure observance de ses traitements ; il devient maître de sa santé}}. Cependant, la démocratisation a pris une forme bien particulière. \ellipse Ainsi, 80 \% des visiteurs du site Doctissimo sont des femmes et disposent d'un pouvoir d'achat non négligeable. Il n'en fallait pas davantage pour transformer ce trait sociologique en stratégie commerciale : le site se définit comme un titre féminin et tire ses revenus de la publicité adressée spécifiquement à ce public.
%
%%\emph{\enquote{C'est la femme qui est responsable de la santé au sein du foyer, analyse Mme Brouchoud. L'une des raisons en est que les femmes ont un suivi médical plus régulier. Elles vont chez le gynécologue tous les ans, elles doivent être suivies quand elles tombent enceintes… Doctissimo a une très grosse communauté de femmes enceintes.}} Ce portrait de la ménagère 2.0 enthousiasme également Bême : \emph{\enquote{La femme, c'est la porte d'entrée de la santé pour toute la famille. C'est elle qui va en parler pour elle, pour ses enfants, pour son mari…}}
%
%%\subsubsection*{Succès des applications mobiles}
%
%\ellipse
%
%%Toutefois, en raison d'un encadrement légal strict des annonces relatives aux médicaments, la publicité dans ce domaine ne s'obtient pas facilement. Et puis, la santé étant généralement perçue comme gratuite en France, \emph{\enquote{personne n'était prêt à payer}} pour l'information qui lui est liée sur Internet, se rappelle Mme Brouchoud. \emph{\enquote{C'est pourquoi nous avons construit un modèle exclusivement basé sur de la publicité et des partenariats avec des annonceurs assez classiques du féminin, notamment dans l'hygiène et la beauté.}}
%Ce modèle économique allait se répercuter sur la production éditoriale. Rapidement perçu au début des années 2000 comme le site de référence français sur la santé, Doctissimo entame une diversification. La direction
%%nous confie une journaliste qui y a travaillé plusieurs années et qui souhaite rester anonyme,
%\ellipse
%\emph{\enquote{a mis l'accent sur d'autres rubriques : psycho, forme, beauté, etc. Les publicités venant surtout des annonceurs beauté-forme, on avait intérêt à se positionner par rapport aux féminins. Par exemple, on comptait L'Oréal parmi les annonceurs}}.
%%Les marques d'automobiles ne dédaignent pas pour autant de présenter leurs derniers modèles à la proverbiale ménagère. Mais, dans le panthéon de Doctissimo, la divinité sévère de la publicité n'occupe qu'une place secondaire. Ici, le maître, c'est Google.
%
%%\emph{\enquote{Nous avons beaucoup travaillé pour comprendre le fonctionnement de l'algorithme. Nous faisions de multiples essais, se souvient M. Alexandre. L'informatique était une matière noble chez Doctissimo. J'avais l'habitude de dire : \enquote{Les gens qui sont à la cave chez \emph{Elle} sont au sommet chez Doctissimo.} L'informaticien était plus important que le journaliste. Des journalistes capables d'écrire un article sur un certain type de régime, il y en avait des milliers ; un informaticien capable de maîtriser le référencement, c'était plus rare. Nous avons beaucoup réfléchi à la façon d'optimiser l'écriture de nos articles pour les rendre \enquote{Google-friendly}.}}
%
%%Ce travail de référencement place Doctissimo en tête du peloton des sites les plus dépendants de Google. Mais capter l'attention ne suffit pas ; il faut aussi la retenir. Un internaute reste en moyenne neuf minutes sur Doctissimo (contre trois minutes sur les autres sites, selon Google Analytics). Là aussi, rien n'est laissé au hasard. \emph{\enquote{L'observation de la navigation révèle rapidement une stratégie de captation des visiteurs au sein d'un territoire éditorial fermé présentant de nombreux liens autoréférentiels : les seules ouvertures offertes sont constituées par les liens vers les annonceurs publicitaires}}, notaient Annelise Touboul et Elizabeth Vercher, maîtresses de conférences à l'université de Lyon, dans une étude menée en 2008.
%
%%\emph{\enquote{Au moment où j'ai vendu Doctissimo à Lagardère, en 2008, nous faisions des millions de bénéfices grâce à la publicité}}, s'enorgueillit M. Alexandre. La valeur de l'entreprise a néanmoins été divisée par dix depuis 2008 ; un effondrement que le fondateur explique par le faible investissement de Lagardère Active dans ses activités médias. Mais le site continue d'engranger des audiences énormes — onze millions de visiteurs uniques en janvier 2019 —, dont plus de 80 \% provient de Google. Résultat : la direction éditoriale de Doctissimo choisit les sujets à traiter en fonction des requêtes les plus fréquentes adressées au moteur de recherche. \emph{\enquote{Si on voit qu'on n'a pas le contenu correspondant, on le fait, explique une journaliste. Après, bien sûr, on propose aussi des sujets en fonction de l'actu, de ce qui se fait ailleurs. On regarde beaucoup ce qu'il se passe sur les forums, car c'est aussi une grosse partie de notre audience : en fonction des questions posées, on crée des sujets, des dossiers, etc.}}

Les communautés des forums [de Doctissimo] régalent particulièrement les annonceurs :
%non seulement elles contribuent au développement du marché de la santé en ligne, mais surtout
\ellipse
elles alimentent les bases de données détaillées sur les profils des utilisateurs. \emph{\enquote{Les \enquote{doctinautes} fournissent au site des informations nombreuses sur eux-mêmes : données socio-démographiques, évolution de la composition du foyer, évolution des habitudes de consommation, centres d'intérêt, avis et opinions (les pathologies et les attitudes à l'égard des problèmes de santé, sur les marques, les produits et les services)}}, exposaient en juin 2016 Mme Lucia Lagarrigue, alors directrice des communautés de Doctissimo chez Lagardère Active, et M. Gilles Achache, de la société Scan-Research, un institut de marketing en ligne. Ces données et messages personnels, traités grâce à des outils d'analyse sémantique, nourrissent la régie publicitaire de Doctissimo, qui propose à ses annonceurs des profils très précis, \emph{\enquote{un niveau d'affinité supplémentaire pour concevoir des opérations spéciales}} — par exemple des publireportages ciblés.

\ellipse

%En 2017, la marque de puériculture Philips Avent finançait ainsi des \enquote{contenus sponsorisés} à destination des femmes enceintes. Doctissimo était alors présenté comme le \emph{\enquote{premier site féminin}}, le \emph{\enquote{premier sur cible femmes 25-49 ans (premier sur cible mamans)}}, avec \emph{\enquote{6,6 millions de membres et 30 000 messages postés par jour}}. Depuis, le nombre d'utilisateurs actifs sur les forums s'est effondré, tombant à 570 000 inscrits actifs. \emph{\enquote{Nous avons fait beaucoup de nettoyage dans le cadre du règlement général sur la protection des données}}, admet Mme Bergamote Bazerolle, responsable du marketing.

%Ce ne sont plus tant les forums et les lettres d'information que les applications mobiles qui captent désormais les lecteurs, et surtout les lectrices : celles baptisées \enquote{Ma grossesse}, \enquote{Mon bébé} et \enquote{Ovulation} cumulent plus d'un million et demi de téléchargements (sur Android). Forte de 120 000 utilisatrices actives par mois, \emph{\enquote{l'application \enquote{Ma grossesse} est conçue autour d'une partie éditoriale, avec une sélection d'articles issus du site. On fait aussi du \enquote{push actu} dès qu'il y a une actualité qui concerne les femmes enceintes}}, détaille Mme Bazerolle. À l'avenir, Doctissimo compte développer des assistants vocaux. \emph{\enquote{Par exemple, la femme pourrait demander directement : \enquote{Est-ce que je peux manger des moules ?} (…) Tout ce qui est hypnose, aide à l'endormissement, ce sont aussi de bons sujets pour les assistants vocaux.}}

Femmes enceintes, \enquote{mamans} : ce public \emph{\enquote{qui surconsomme}}, selon [la responsable du marketing de Doctissimo], constitue également une mine d'informations. Car les utilisatrices livrent à ces applications quantité de données personnelles susceptibles d'électriser les annonceurs : pour profiter du service, il faut indiquer sa date d'accouchement, son poids, la date de ses dernières règles, etc. Couplés aux données laissées par les internautes lors de leur inscription à des lettres d'information ou aux forums, aux messages postés sur les forums, ces éléments dessinent des profils toujours plus précis, et donc toujours plus rentables.

Le rachat de Doctissimo par TF1 s'explique-t-il par le désir de mettre la main sur ces fichiers ? Avec MyTF1, qui oblige les internautes à s'enregistrer pour accéder aux vidéos, le groupe de M. Martin Bouygues possède déjà une base de données de vingt-trois millions d'internautes enregistrés. \emph{\enquote{Notre objectif est en effet de valoriser au mieux cette data, en apportant côté utilisateurs des services plus personnalisés, à forte valeur ajoutée, et, côté annonceurs, la possibilité de proposer des campagnes ciblées. Grâce à cette base de données très qualifiée, nous pouvons faire évoluer notre produit [Doctissimo] en fonction des attentes et des habitudes de nos communautés}}, nous explique, dans le langage fleuri des stratèges, M. Nicolas Capuron, directeur des nouveaux business digitaux du groupe.

%Cette acquisition permet également à TF1 d'accroître ses audiences sur le web, tout comme celle du groupe Aufeminin, en avril 2018, avait propulsé l'entreprise de M. Bouygues dans le classement des dix groupes les plus consultés en ligne. TF1 a par ailleurs acquis la société Gamned !, spécialisée dans la vente et l'achat automatiques d'espaces publicitaires, afin de \emph{\enquote{valoriser les données de ses principales entités (Aufeminin, Doctissimo, Les Numériques, Marmiton…)}}, selon son communiqué.
\ellipse

Président-directeur général du groupe TF1, M. Gilles Pélisson a expliqué le projet humaniste et culturel qui anime les dirigeants de la première chaîne française : \emph{\enquote{Nous pourrons ainsi mettre à disposition de nos annonceurs des capacités de ciblage plus fines et enrichir le dialogue de nos partenaires avec les internautes au-delà de nos supports.}} Il y a tant de manières de dire qu'on vend du temps de cerveau disponible…
\end{mdframed}

\subsection*{Paragraphe argumenté}

\noindent Un paragrahe argumenté est composé des quatre étapes suivantes, en une dizaine de lignes environ.

\begin{enumerate}
\item Annonce de l'idée de l'argument (prendre position).
\item Explication (préciser, expliquer la position prise) : \enquote{En effet…}.
\item Illustration (citer les documents) : \enquote{Ainsi…}.
\item Conclusion (énoncer à nouveau la position prise) : \enquote{Donc…}.
\end{enumerate}

\cuthere

\subsection*{Barème (à découper et à coller sur votre copie)}

\noindent Chacun des items suivants est noté sur un point.

\begin{itemize}[$\ldots/1$]
\item Les quatre étapes du paragraphe argumenté sont présentes.
\item L'argumentation est cohérente.
\item L'argumentation est pertinente.
\item Le document est correctement cité.
\end{itemize}

\end{document}
