%%%%%%%%%%%%%%%%%%%%%%%%%%%%%%%%%%%%%%%%%%%%%%%%%%%%%%%%%%%%%%%%%%%%%%%%%%%%%%%%
% Copyright 2020 Louis Paternault --- http://snt.ababsurdo.fr
%
% Publié sous licence Creative Commons Attribution-ShareAlike 4.0 International (CC BY-SA 4.0)
% http://creativecommons.org/licenses/by-sa/4.0/deed.fr
%%%%%%%%%%%%%%%%%%%%%%%%%%%%%%%%%%%%%%%%%%%%%%%%%%%%%%%%%%%%%%%%%%%%%%%%%%%%%%%%

% Pour compiler :
%$ lualatex $basename

\documentclass[12pt]{article}

\usepackage[a5paper, margin=1cm]{geometry}

\usepackage{textcomp}
\usepackage[francais]{babel}
\usepackage[shortlabels]{enumitem}
\usepackage{hyperref}
\hypersetup{
  unicode=true,
  urlcolor=cyan,
  pdfauthor={Louis Paternault},
  pdfproducer={© 2020 Louis Paternault — CC-BY-SA-4.0 — http://snt.ababsurdo.fr},
  hidelinks,
}

\usepackage{tabularx}
\usepackage{multirow}
\usepackage{graphicx}

\pagestyle{empty}

\setlength{\parindent}{0pt}

\begin{document}

Noms : \dotfill

\bigskip

Rôle : \textbf{Équipier/Équipière}

\bigskip

Après avoir préparé vos arguments en équipe, vous allez choisir deux personnes parmi vous (les orateurs ou oratrices) qui vont débattre. Les autres co-équipiers \emph{n'interviennent pas} pendant le débat, sauf si les orateurs et oratrices demandent un temps mort (pendant lequel toute l'équipe peut se concerter durant une minute, mais pas changer les orateurs et oratrices).

Le but des orateurs et oratrices est de présenter des arguments pour \emph{convaincre} l'équipe adverse et le jury.

Voici les critères d'évaluation que les membres du jury vont utiliser.

\bigskip
\begin{tabularx}{\textwidth}{c|X|X}
  &\multicolumn{1}{c|}{\textbf{Forme}}&\multicolumn{1}{c|}{\textbf{Fond}}\\
  \hline
  \multirow{5}{*}{\rotatebox[origin=c]{90}{\textbf{Positif}}}
  & Clarté des propos & Nombre d'arguments \\
  & Capacité à écouter & Capacité à contre-argumenter  \\
  & Capacité à convaincre (posture, ton, regard, gestes) & Capacité à réorienter/relancer le débat \\
  & Temps morts demandés au bon moment & Qualité des arguments (maîtrise, précision) \\
  & Capacité à reprendre la parole & \\
  \hline
  \multirow{5}{*}{\rotatebox[origin=c]{90}{\textbf{Négatif}}}
  & Agressivité
  & Propos hors sujet \\
  & Couper la parole
  & Répétitions (enferme le débat) \\
  & Rester silencieux / être en retrait
  & Affirmations gratuites (sans preuves) \\
  & Monopoliser la parole
  & Contradictions / erreurs \\
  & Les co-équipiers parlent pendant les rounds
  & \\
\end{tabularx}

\vfill
{\scriptsize
\emph{Tableau recopié depuis :}\\
\url{https://sesame.apses.org/index.php?option=com_content&view=article&id=84&Itemid=235}}

\end{document}
