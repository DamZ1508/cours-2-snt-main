%%%%%%%%%%%%%%%%%%%%%%%%%%%%%%%%%%%%%%%%%%%%%%%%%%%%%%%%%%%%%%%%%%%%%%%%%%%%%%%%
% Copyright 2020 Louis Paternault --- http://snt.ababsurdo.fr
%
% Publié sous licence Creative Commons Attribution-ShareAlike 4.0 International (CC BY-SA 4.0)
% http://creativecommons.org/licenses/by-sa/4.0/deed.fr
%%%%%%%%%%%%%%%%%%%%%%%%%%%%%%%%%%%%%%%%%%%%%%%%%%%%%%%%%%%%%%%%%%%%%%%%%%%%%%%%

% Pour compiler :
%$ lualatex $basename

\documentclass[12pt]{article}

\usepackage[a6paper, landscape, margin=.5cm]{geometry}

\usepackage{textcomp}
\usepackage[francais]{babel}
\usepackage[shortlabels]{enumitem}
\usepackage{tabularx}
\usepackage{hyperref}
\hypersetup{
  unicode=true,
  urlcolor=cyan,
  pdfauthor={Louis Paternault},
  pdfproducer={© 2020 Louis Paternault — CC-BY-SA-4.0 — http://snt.ababsurdo.fr},
  hidelinks,
}

\pagestyle{empty}

\setlength{\parindent}{0pt}

\newcommand{\debut}[1]{%
  Nom : \dotfill

  Rôle : \textbf{Jury}

  Votre rôle est d'évaluer les équipes, en utilisant la grille suivante (d'autres membres du jury ont d'autres grilles).

  Comptez les #1 de chacune des deux équipes en utilisant la grille suivante.
}
\newenvironment{tableau}[1]{

  \bigskip

  \tabularx{\textwidth}{|p{5cm}||X|X|}
  \hline
  \multicolumn{1}{|c||}{\textbf{#1}} &
  \multicolumn{1}{c||}{Équipe 1} &
  \multicolumn{1}{c|}{Équipe 2} \\
  \hline
  \hline
}{
  \hline
\endtabularx
}

\newcommand{\fin}{\scriptsize
  \vfill
  \emph{Tableau recopié depuis :}\\
  \url{https://sesame.apses.org/index.php?option=com_content&view=article&id=84&Itemid=235}
}

\begin{document}

\debut{points}
\begin{tableau}{Forme --- Positif}
  Clarté des propos &&\\
  \hline
  Capacité à écouter &&\\
  \hline
  Capacité à convaincre (posture, ton, regard, gestes) &&\\
  \hline
  Temps morts demandés au bon moment &&\\
  \hline
  Capacité à reprendre la parole &&\\
\end{tableau}
\fin
\newpage

\debut{points}
\begin{tableau}{Fond --- Positif}
Nombre d'arguments &&\\
\hline
Capacité à contre-argumenter &&\\
\hline
Capacité à réorienter/relancer le débat &&\\
\hline
Qualité des arguments (maîtrise, précision) &&\\
\end{tableau}
\fin
\newpage

\debut{pénalités}
\begin{tableau}{Forme --- Négatif}
Agressivité &&\\
\hline
Couper la parole &&\\
\hline
Rester silencieux / être en retrait &&\\
\hline
Monopoliser la parole &&\\
\hline
Les co-équipiers parlent pendant les rounds &&\\
\end{tableau}
\fin
\newpage

\debut{pénalités}
\begin{tableau}{Fond --- Négatif}
Propos hors sujet &&\\
\hline
Répétitions (enferme le débat) &&\\
\hline
Affirmations gratuites (sans preuves) &&\\
\hline
Contradictions / erreurs &&\\
\end{tableau}
\fin

\end{document}
