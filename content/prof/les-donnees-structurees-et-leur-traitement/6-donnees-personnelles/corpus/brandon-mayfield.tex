\section{The terrifying surveillance case of Brandon Mayfield}
\source{Matthew Harwood, Aljazeera America, 8/02/2014\footnote{\url{http://america.aljazeera.com/opinions/2014/2/the-terrifying-surveillancecaseofbrandonmayfield.html}}, \emph{traduit par le Framablog :} L'affaire Brandon Mayfield : une surveillance terrifiante, Framablog, le 12/02/2014\footnote{\url{https://framablog.org/2014/02/12/le-cas-brandon-mayfield/}}.}

Le 11 mars 2004 à Madrid, des terroristes proches de la mouvance d'Al-Qaïda ont coordonné un attentat à la bombe sur plusieurs trains de banlieue durant l'heure de pointe matinale. 193 personnes furent tuées et environ 1 800 furent blessées. Deux empreintes digitales partielles découvertes sur un sac de détonateurs au cours de l'enquête par la Police Nationale Espagnole (PNE) furent partagées avec le FBI par le biais d'Interpol. Les deux empreintes furent entrées dans la base de données du FBI, qui retourna vingt concordances possibles pour l'une d'entre elles : sur ces vingt concordances, l'une appartenait à Brandon Mayfield.
[…]
Ses empreintes étaient répertoriées dans le système du FBI parce qu'il avait fait son service militaire mais aussi parce qu'il avait été arrêté sur un malentendu vingt ans auparavant. Les charges avaient ensuite été abandonnées.

[…] % Il se trouvait que l'empreinte de Mayfield ne concordait pas exactement avec l'empreinte laissée sur le sac de détonateurs. Malgré cela, les spécialistes des empreintes au FBI fournirent des justifications aux différences, selon un rapport du Bureau de l'inspecteur général du département de la Justice (OIG). […]
Certains détails de la vie de l'avocat ont convaincu les agents qu'ils tenaient leur homme. Mayfield s'était converti à l'Islam après avoir rencontré sa femme, une égyptienne. Il avait offert son aide juridique sur une affaire de garde d'enfant à l'un des \enquote{Portland Seven}, un groupe d'hommes qui avait essayé d'aller en Afghanistan afin de combattre pour Al-Qaïda et les Talibans contre les États-Unis et leurs forces alliées. Il fréquentait aussi la même mosquée que les militants. […] % À la suite des événements du 11 Septembre, ces innocentes associations et relations, quoique approximatives, étaient devenues pour les enquêteurs des preuves que Mayfield n'était pas un bon citoyen américain, mais un terroriste sanguinaire déterminé à détruire l'Occident.

Des agents du FBI pénétrèrent par effraction dans la maison de Mayfield et dans son cabinet d'avocat.
Ils fouillèrent dans des documents protégés par le secret professionnel entre un avocat et son client, ils mirent sur écoute ses téléphones, ils analysèrent sa comptabilité et son historique de navigation Internet, ils fouillèrent même ses poubelles.
Ils le suivirent dans tous ses déplacements.
% Malgré tout cela, le FBI ne trouva pas le moindre indice qui puisse le relier aux événements de Madrid. Ils trouvèrent cependant, des recherches Internet sur des vols vers l'Espagne et découvrirent qu'il avait pris une fois des leçons de pilotage d'avion. Pour ces agents du FBI, d'ores et déjà convaincus de sa culpabilité, tout cela était la preuve que Mayfield était un terroriste en puissance.

[…]

Pensant que leur couverture avait sauté, les agents du FBI placèrent Mayfield en détention comme témoin matériel dans l'attentat de Madrid, au motif qu'ils craignaient un risque de fuite.
[…] % Ils ne pouvaient pas l'arrêter car malgré leur surveillance intrusive ils n'avaient toujours aucune preuve tangible d'aucun crime.
Il passa deux semaines en prison, terrifié à l'idée que ses codétenus apprennent qu'il était impliqué d'une manière ou d'une autre dans les attentats de Madrid et qu'ils ne l'agressent.

[…]

La seule raison pour laquelle Mayfield est un homme libre aujourd'hui, c'est que la police espagnole a répété à plusieurs reprises au FBI que l'empreinte récupérée sur le sac de détonateurs ne correspondait pas à celles de Mayfield. 
