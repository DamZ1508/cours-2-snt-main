\section{Marc L***}
\source{Raphaël Meltz, Le Tigre \no28, novembre-décembre 2008\footnote{\url{http://le-tigre.net/marc-L.html}}.}

Bon annniversaire, Marc. Le 5 décembre 2008, tu fêteras tes vingt-neuf ans. Tu permets qu'on se tutoie, Marc ? Tu ne me connais pas, c'est vrai. Mais moi, je te connais très bien. C'est sur toi qu'est tombée la (mal)chance d'être le premier portrait Google du Tigre. Une rubrique toute simple : on prend un anonyme et on raconte sa vie grâce à toutes les traces qu'il a laissées, volontairement ou non sur Internet.
%Comment ça, un message se cache derrière l'idée de cette rubrique ? Évidemment : l'idée qu'on ne fait pas vraiment attention aux informations privées disponibles sur Internet, et que, une fois synthétisées, elles prennent soudain un relief inquiétant. Mais sache que j'ai plongé dans ta vie sans arrière-pensée : j'adore rencontrer des inconnus. Je préfère te prévenir : ce sera violemment impudique, à l'opposé de tout ce qu'on défend dans Le Tigre. Mais c'est pour la bonne cause ; et puis, après tout, c'est de ta faute : tu n'avais qu'à faire attention.

[…]

Alors, Marc. Belle gueule, les cheveux mi-longs, le visage fin et de grands yeux curieux. Je parle de la photo prise au Starbuck's Café de Montréal, lors de ton voyage au Canada, avec Helena et Jose, le 5 août 2008. La soirée avait l'air sympa, comme d'ailleurs tout le week-end que vous avez passé à Vancouver. J'aime particulièrement cette série, parce que Jose a fait des photos, et ça me permet de te voir plus souvent. Vous avez loué un scooter, vous êtes allés au bord de la mer, mais vous ne vous êtes pas baignés, juste traîné sur la plage. En tout, tu as passé un mois au Canada. Au début tu étais seul, à l'hôtel Central, à Montréal (série de photos \enquote{autour de mon hôtel}). Tu étais là-bas pour le travail. Le travail ? Tu es assistant au \enquote{service d'architecture intérieur}, dans un gros cabinet d'architectes, LBA, depuis septembre dernier (Facebook, rubrique Profil). Le cabinet a des succursales dans plusieurs villes, et a priori tu dois travailler dans la succursale de Pessac, dans la banlieue de Bordeaux. Ça, je l'ai trouvé par déduction, vu que tu traînes souvent à l'Utopia (cinéma et café bordelais) ou à Arcachon. Donc à Montréal, tu étais dans un bureau avec Steven, Philipp, Peter, en train de travailler sur des plans d'architectes, devant deux ordinateurs, un fixe et un portable. […] Le 21 août, c'est Steven qui t'a accompagné à l'aéroport. Retour en France, où t'attendait un mariage (Juliette et Dominique), puis, la semaine suivante, le baptême de ta nièce, Lola, la petite sœur de Luc (qui fait des têtes rigolotes avec ses grosses lunettes), à Libourne.

Revenons à toi. Tu es célibataire et hétérosexuel (Facebook). Au printemps 2008, tu as eu une histoire avec Claudia R***, qui travaille au Centre culturel franco-autrichien de Bordeaux (je ne l'ai pas retrouvée tout de suite, à cause du caractère ü qu'il faut écrire ue pour Google). En tout cas, je confirme, elle est charmante, petits seins, cheveux courts, jolies jambes. Tu nous donnes l'adresse de ses parents, boulevard V*** à Bordeaux. Vous avez joué aux boules à Arcachon, et il y avait aussi Lukas T***, qui est le collègue de Claudia au Centre Culturel. Fin mai, il n'y a que quatre photos, anodines, de ton passage dans le petit appartement de Claudia (comme si tu voulais nous cacher quelque chose) et une autre, quelques jours plus tard, plus révélatrice, prise par Claudia elle-même, chez elle : on reconnaît son lit, et c'est toi qui es couché dessus. Habillé, tout de même. Sur une autre, tu te brosses les dents. C'est le 31 mai : deux jours plus tôt, vous étiez chez Lukas \enquote{pour fêter les sous de la CAF} (une fête assez sage, mais Lukas s'est mis au piano pour chanter des chansons en allemand, tout le monde a bien ri, vidéo sur Flickr). Ce 31 mai, vous avez une façon de vous enlacer qui ne laisse que peu de doutes. Et le 22 juin, cette fois c'est sûr, vous vous tenez par la main lors d'une petite promenade au Cap-Ferret. C'est la dernière fois que j'ai eu des nouvelles de Claudia. Note bien que j'ai son numéro au travail (offre d'emploi pour un poste d'assistant pédagogique au Centre culturel, elle s'occupe du recrutement), je pourrais l'appeler. Mais pour raconter une séparation, même Internet a des limites. Avant Claudia, tu étais avec Jennifer (ça a duré au moins deux ans), qui s'intéressait à l'art contemporain (vous avez visité ensemble Beaubourg puis tu l'as emmenée au concert de Madonna à Bercy). Elle a habité successivement Angers puis Metz, son chat s'appelle Lula, et, physiquement, elle a un peu le même genre que Claudia. À l'été 2006, vous êtes partis dans un camping à Pornic, dans une Golf blanche. La côte Atlantique, puis la Bretagne intérieure. Tu avais les cheveux courts, à l'époque, ça t'allait moins bien.

On n'a pas parlé de musique. À la fin des années 1990, tu as participé au groupe Punk, à l'époque où tu habitais Mérignac (à quelques kilomètres de Bordeaux). Il reste quelques traces de son existence, sur ton Flicker bien sûr mais aussi dans les archives Google de la presse locale. Tu sais quoi ? C'est là que j'ai trouvé ton numéro de portable : 06 83 36 ** **. Je voulais vérifier si tu avais gardé le même numéro depuis 2002. Je t'ai appelé, tu as dit : \enquote{Allô ?}, j'ai dit : \enquote{Marc ?}, tu as dit : \enquote{C'est qui ?}, j'ai raccroché. Voilà : j'ai ton portable. […]

[…]

Je pense à l'année 1998, il y a dix ans, quand tout le monde fantasmait déjà sur la puissance d'Internet. Le Marc L*** de l'époque, je n'aurais sans doute rien ou presque rien trouvé sur lui. Là, Marc, j'ai trouvé tout ce que je voulais sur toi. J'imagine ton quotidien, ta vie de jeune salarié futur architecte d'intérieur, ton plaisir encore à faire de la musique avec tes potes à Bordeaux, tes voyages à l'autre bout du monde, ta future petite copine (je parie qu'elle aura les cheveux courts). Mais il me manque une chose : ton adresse. Dans ces temps dématérialisés, où mails et téléphones portables tiennent lieu de domiciliation, ça me pose un petit problème : comment je fais pour t'envoyer Le Tigre ? Je sais que tu es avenue F***, mais il me manque le numéro, et tu n'es pas dans les pages jaunes. Cela dit, je peux m'en passer. Il suffit que je ne te l'envoie pas, ton portrait : après tout, tu la connais déjà, ta vie.
