\section{From grainy CCTV to a positive ID: Recognising the benefits of surveillance}
\source{Rob Hastings, The Independant, 01/01/2013\footnote{\url{https://www.independent.co.uk/news/uk/home-news/from-grainy-cctv-to-a-positive-id-recognising-the-benefits-of-surveillance-8434644.html}}, traduit par Louis Paternault.}

Le professeur Mark nikon \enquote{a eu quelques échanges tendus} à l'époque avec les groupes de protection des libertés publiques. Expert mondial dans la reconnaissance des personnes par vidéosurveillance en utilisant la biométrie --- la bête noire de tous les militants anti-surveillance --- il sait très bien ce qu'ils pensent de son travail.

\enquote{Ils disent que nous violons leur vie privée}, dit-il. \enquote{Je ne pense pas que leur liberté soit en danger.} Les techniques qu'il a inventées \enquote{ont été utilisées pour attraper des meurtrier --- et ça me convient très bien}.

Grâce au travail du professeur Nixon et du docteur John Carter à l'université de Southampton, il est devenu extrèmement facile pour les autorité de tous nous surveiller. Des recherches à la \emph{School of Electronics and Computer Science} (école d'électronique et d'informatique) --- financées entre autres par le Pentagone et le ministère de la défense --- permettent de trouver et d'identifier de plus en plus précisément des criminels dans des images de vidéosurveillance.

[…]

Cela a permis de résoudre plusieurs crimes en Grande-Bretagne, et a aidé à arrêté l'assasin de l'ancienne ministre Anna Lindh en Suède en 2003.

Le potentiel de tels systèmes de surveillance est énorme. Mais les risques d'atteinte à la vie privée aussi, préviennent les militants.

[…]

Mais le professeur Nixon n'est pas d'accord. \enquote{Les inquiétudes par rapport à la vie privée sont très vieilles. Si vous vous intéressez à l'histoire des cartes, vous verrez que les gens ne voulaient pas que des salauds insolents cartographient leurs terres. Les gens ont toujours été méfiants de ce qu'ils ne connaissent pas, mais la plupart diraient que cela ne les dérangerait pas que la police sache plein de choses sur eux si cela permettait d'éliminer des menaces sérieuses. […]}

La demande [en loi encadrant la vidéosurveillance] va probablement ne faire qu'augmenter. Bientôt, l'intelligence artificielle sera capable d'alerter automatiquement les personnels de sécurité sur des comportement suspects avant même que la personne ait fait quoi que ce soit d'illégal.
