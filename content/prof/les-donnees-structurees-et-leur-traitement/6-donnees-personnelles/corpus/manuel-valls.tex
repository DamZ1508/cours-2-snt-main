\section{Projet de loi sur le renseignement : Conférence de presse de Manuel Valls}
\source{Le 19/03/2015\footnote{\url{https://www.dailymotion.com/video/x2kjre7}}.}

\begin{em}
  Contexte : Dans cette conférence de presse, pour lutter contre le terrorisme (en particulier les attentats djihadistes), le premier ministre Manuel Valls décrit une loi augmentant les capacités des services de renseignement à surveiller les communications électroniques (internet, téléphone) des citoyens français.
\end{em}

Le second objectif est de garantir la protection des libertés publiques.
Aucune mesure de surveillance ne pourra être effectuée sans autorisation préalable et sans contrôle indépendant.
Cette loi sera aussi protectrice des citoyens car les limites de ce qu'il est possible de faire dans un état de droit seront gravées dans le marbre : il n'y aura plus de zone grise.

[…]

S'agissant des mesures, le projet de loi codifie les différents types de surveillance et détermine les règles précises qui seront applicables à chacune d'elle : interception de sécurité, accès aux données de connexion, géolocalisation en temps réel […], intrusion informatique pour contourner les effets de cryptage, […].
En résumé, plus les techniques touchent la vie privée, plus les contraintes sont fortes et les durées autorisées limitées.

[…]

Mesdames et messieurs, face à l'accroissement de la menace djihadiste, il faut donc renforcer encore l'efficacité de la surveillance des terroristes.
%Aujourd'hui, seul un individu sur deux qui arrive en Syrie a été préalablement détecté avant son départ.

[…]

Certaines inquiétudes, notamment chez les acteurs du numérique, s'expriment à ce sujet. […]
Je vais vous le dire de manière très claire : il ne s'agit en aucun cas de mettre en œuvre des moyens d'exception ou une surveillance généralisée des citoyens. Le projet de loi prévoit expressément que cette surveillance renforcée concernera les communications des seuls terroristes.
Cela démontre bien qu'il n'y aura aucune surveillance de masse. Le projet de loi l'interdit.

[…]

Pour identifier les moyens de communication des individus qui cherchent en permanence à dissimuler leurs échanges, les possibilités accordées en la matière aux services de renseignement seront limitées, infiniment plus réduites que dans le cas de la police judiciaire.
Seule l'identification et l'aide aux surveillances de terrain seront autorisées. Il n'y aura en aucun cas aspiration massive des données personnelles, comme j'ai pu le lire dans certains journaux au cours des derniers jours.
