\section{Alicem, la première solution d'identité numérique régalienne sécurisée}
\source{Ministère de l'intérieur, 16 décembre 2019\footnote{\url{https://www.interieur.gouv.fr/Actualites/L-actu-du-Ministere/Alicem-la-premiere-solution-d-identite-numerique-regalienne-securisee}}.}

\begin{em}
Contexte : Alicem est une application pour mobile proposée par le ministère de l'intérieur permettant de s'authentifier par reconnaissance faciale.
\end{em}

Les données personnelles sont-elles conservées et sécurisées ?

\begin{itemize}
\item Les données extraites du titre d'identité sont vérifiées lors de l'inscription mais ces dernières ne sont stockées que sur le smartphone de l'utilisateur sous son contrôle exclusif et protégées par un chiffrement.
\item Alicem n'a pas accès aux historiques de transactions grâce à la séparation garantie par la plateforme \enquote{FranceConnect}, qui anonymise les fournisseurs de service auxquelles sont transmises les données.
\item Le décret qui réglemente Alicem contient des dispositions très strictes sur la gestion des données.
\item Les données ne font l'objet d'aucune utilisation pour d'autres objectifs que l'authentification électronique et l'accès à des services en ligne par Alicem. Elles ne sont pas transmises à des tiers.
\end{itemize}
