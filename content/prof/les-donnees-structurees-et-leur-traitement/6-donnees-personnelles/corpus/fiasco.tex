\section{Le fiasco de la reconnaissance faciale testée par la police de Londres}
\source{Samuel Khan, Le Figaro, 5 mars 2020\footnote{\url{https://www.lefigaro.fr/secteur/high-tech/le-fiasco-de-la-reconnaissance-faciale-testee-par-la-police-de-londres-20200305}}.}

%Si vous êtes passé par le très fréquenté quartier d'Oxford Circus vendredi dernier, votre visage a peut-être été photographié et comparé à une liste de personnes recherchées par la police londonienne.
Après plusieurs mois de tests, Scotland Yard [la police Britannique] a procédé aux premières opérations de surveillance fondée sur la reconnaissance faciale.
Une camionnette équipée de cette technologie qui permet de scanner et de reconnaître un visage à la volée a été installée à la sortie du métro [d'Oxford Circus, à Londres].
[…]
%Si l'opération était signalée par des affiches, il était impossible de s'y opposer, sauf à changer de trottoir.
Des milliers de personnes ont ainsi eu leur visage filmé sans leur consentement.
Mais les résultats sont loin d'être à la hauteur des promesses de la Met Police, pour qui ce système doit permettre de \enquote{combattre les crimes, la violence et l'exploitation sexuelle des enfants}.

Sur les \numprint{8600} personnes dont le visage a été filmé dans la journée du 27 février, huit ont été reconnues comme faisant partie d'un fichier de personnes recherchées spécifiquement pour des crimes violents.
Problème : sept d'entre elles l'ont été à tort, donnant lieu à plusieurs contrôles non justifiés.
Soit un taux d'erreur de près de 90 \%, bien loin des 70 \% de précision promis par la police au début des tests.
[…]
%L'association de défense des libertés Big Brother Watch s'est émue de l'émergence d'un \enquote{état de surveillance orwellien}.
%Avec \numprint{627 000} caméras installées dans ses rues, Londres est la sixième ville la plus surveillée du monde, notamment derrière Shanghaï.
