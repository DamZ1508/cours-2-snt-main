\section{Non, je n'ai rien à cacher}
\source{Ploum, 21 novembre 2012\footnote{\url{https://ploum.net/rien-a-cacher/}}.}

Adolescent dans une grande école catholique, je me fais un jour approcher par un condisciple.
--- Lio, il faut que je montre un truc trop drôle !

Ce camarade me révèle qu'il a trouvé, dans une revue porno, une photo ressemblant fortement à un de nos éducateurs. Intrigué, je demande bien sûr à voir la photo en question. Publiée dans la rubrique \enquote{courrier des lecteurs}, elle représente un homme nu en érection. Contrairement aux autres photos de cette rubrique, le visage n'est pas flouté. Et la ressemblance est, il est vrai, frappante.

Éclatant de rire, nous avons vite fait de nous adjoindre une petite troupe goguenarde autour de la photo. Je remarque alors une chevalière très particulière et un pendentif en or au cou de notre exhibitionniste.

Ni une ni deux, la petite troupe décide de passer \enquote{discrètement} devant le bureau des éducateurs pour vérifier et, stupeur, notre éducateur porte la même chevalière, le même pendentif. Il n'y a donc plus aucun doute.

[…]

De mon côté, intrépide et inconscient, je lui demande de me découper la photo et la fait passer sous le manteau dans l'école. C'est rigolo. Les élèves jasent.

Le lendemain, l'éducateur n'est pas là. Il ne reviendra jamais.

Cet éducateur avait-il quelque chose à se reprocher ? Non, il échangeait une photo où il apparaissait nu avec un public majeur consentant et demandeur. C'était tout à fait légal et on ne peut lui reprocher cela.

Par contre, le magazine est arrivé dans les mains d'un lecteur non-majeur. La personne ayant permis cela est donc coupable car la photo, bien que parfaitement légale, mine l'autorité de l'éducateur. De plus, elle va à l'encontre des valeurs morales affichées par l'employeur. Deux raisons qui font qu'il était impossible de garder l'éducateur en poste.

Il est donc important de souligner un point : le problème n'est pas que l'éducateur aie posé pour des photos pornographiques ni même qu'elles aient été publiées mais bien que les élèves subordonnés à l'éducateur en prirent connaissance. Ce n'est pas le fait ni l'information qui pose problème mais bien que certaines personnes particulières aient accès à cette information.

La phrase \enquote{Celui qui n'a rien à se reprocher n'a rien à cacher} est donc fausse car ce n'est pas vous qui choisissez ce que vous vous reprochez. C'est le public qui a tout pouvoir pour décider ce qu'il va décider de vous reprocher. Afin d'illustrer la nécessité de la vie privée, on prend souvent l'exemple du régime totalitaire qui contrôle les citoyens. […]
