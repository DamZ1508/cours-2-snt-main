\section{My reaction to Eric Schmidt}
\source{Bruce Schneier, 9/12/2009\footnote{\url{https://www.schneier.com/blog/archives/2009/12/my_reaction_to.html}}, \emph{traduit par Tristan Nicot}, Dérapage d'Eric Schmidt, de Google, 11/12/2009\footnote{\url{http://standblog.org/blog/post/2009/12/11/Dérapage-d-Eric-Schmidt-de-Google}}.}

La notion de vie privée nous protège de ceux qui ont le pouvoir, même si nous ne faisons rien de mal au moment où nous sommes surveillés. Nous ne faisons rien de mal quand nous faisons l'amour ou allons aux toilettes. Nous ne cachons riens délibérément quand nous cherchons des endroits tranquilles pour réfléchir ou discuter. Nous tenons des journaux intimes, chantons seuls sous la douche, écrivons des lettres à des amoureux secrets pour ensuite les brûler. La vie privée est un besoin humain de base.

[…] Si nous sommes observés en toute occasion, nous sommes en permanence menacés de correction, de jugement, de critique, y compris même le plagiat de nous-même. Nous devenons des enfants, emprisonnés par les yeux qui nous surveillent, craignant en permanence que --- maintenant ou plus tard --- les traces que nous laissons nous rattraperont, par la faute d'une autorité quelle qu'elle soit qui porte maintenant son attention sur des actes qui étaient à l'époque innocents et privés. Nous perdons notre individualité, parce que tout ce que nous faisons est observable et enregistrable. […]

Voici la perte de liberté que nous risquons quand notre vie privée nous est retirée. C'est la vie dans l'ex-Allemagne de l'Est ou dans l'Irak de Saddam Hussein. Mais c'est aussi notre futur si nous autorisons l'intrusion de ces yeux insistants dans nos vies personnelles et privées.

Trop souvent on voit surgir le débat dans le sens ``sécurité contre vie privée''. Le choix est en fait liberté contre contrôle. La tyrannie, qu'elle provienne de la menace physique d'une entité extérieure ou de la surveillance constante de l'autorité locale, est toujours la tyrannie. La liberté, c'est la sécurité sans l'intrusion, la sécurité avec en plus la vie privée. La surveillance omniprésente par la police est la définition même d'un état policier. Et c'est pour cela qu'il faut soutenir le respect de la vie privée même quand on n'a rien à cacher.
