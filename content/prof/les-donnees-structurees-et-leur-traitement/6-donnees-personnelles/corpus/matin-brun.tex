\section{Matin brun}
\source{Franck Pavloff, édition Cheyne, 2008.}
\begin{em}
Dans ce roman, le lecteur suit deux amis (le narrateur et Charlie) dans un pays où le gouvernement a interdit tous les chats sauf les bruns. Puis tous les chiens sauf les bruns. Puis la liste des choses interdites (sauf les brunes) s'allonge.
\end{em}

Quelque temps après, c'est moi qui avais appris à Charlie que le Quotidien de la ville ne paraîtrait plus. Il en était resté sur le cul : le journal qu'il ouvrait tous les matins en prenant son café crème ! […]
\begin{itemize}
  \item Pas un jour sans s'attaquer à cette mesure nationale. Ils allaient jusqu'à remettre en cause les résultats des scientifiques. % Les lecteurs ne savaient plus ce qu'il fallait penser, certains même commençaient à cacher leur clébard !
  \item À trop jouer avec le feu\ldots
  \item Comme tu dis, le journal a fini par se faire interdire.
\end{itemize}
[…]
Après ça avait été au tour des livres de la bibliothèque, une histoire pas très claire, encore.
Les maisons d'édition qui faisaient partie du même groupe financier que le Quotidien de la ville, étaient poursuivies en justice et leurs livres interdits de séjour sur les rayons des bibliothèques.

[…]

J'allais chez Charlie.
[…] Et là, surprise totale : la porte de son appart avait volé en éclats, et deux miliciens plantés sur le palier faisaient circuler les curieux.
J'ai fait semblant d'aller dans les étages du dessus et je suis redescendu par l'ascenseur.
En bas, les gens parlaient à mi-voix.
\begin{itemize}
  \item Pourtant son chien était un vrai brun, on l'a bien vu, nous !
  \item Oui, mais à ce qu'ils disent, c'est que avant, il en avait un noir, pas un brun. Un noir.
  \item Avant ?
  \item Oui, avant. Le délit maintenant, c'est aussi d'en avoir eu un qui n'aurait pas été brun. Et ça, c'est pas difficile à savoir, il suffit de demander au voisin.
\end{itemize}
J'ai pressé le pas. Une coulée de sueur trempait ma chemise. Si en avoir eu un avant était un délit, j'étais bon pour la milice. Tout le monde dans mon immeuble savait qu'avant j'avais eu un chat noir et blanc. Avant ! Ça alors, je n'y aurais jamais pensé ! Ce matin, Radio brune a confirmé la nouvelle. […]
\enquote{Avoir eu un chien ou un chat non conforme, à quelque époque que ce soit, est un délit.} Le speaker a même ajouté \enquote{Injure à l'État national} 

Et j'ai bien noté la suite. Même si on n'a pas eu personnellement un chien ou un chat non conforme, mais que quelqu'un de sa famille, un père, un frère, une cousine par exemple, en a possédé un, ne serait ce qu'une fois dans sa vie, on risque soi-même de graves ennuis.

Je ne sais pas où ils ont amené Charlie. Là, ils exagèrent. C'est de la folie. Et moi qui me croyais tranquille pour un bout de temps avec mon chat brun.
