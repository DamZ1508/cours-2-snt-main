\section{Google chief: Only miscreants worry about net privacy}
\source{Cade Metz, The Register, 7/12/2019\footnote{\url{https://www.theregister.co.uk/2009/12/07/schmidt_on_privacy}}, traduit par Louis Paternault.}

\section*{Le patron de Google : Seuls les vauriens s'inquiètent de la vie privée sur internet}

Si vous êtes inquiets que Google conserve vos données personnelles, alors vous faites sans doute quelque chose que vous ne devriez pas. Du moins, c'est ce que prétend Eric Schmidt, le directeur général de Google.

\enquote{Si vous ne voulez pas que qui que se soit soit au courant de quelque chose, peut-être que vous ne devriez tout simplement pas le faire,} a dit Schmidt […].

Mais la vraie nouvelle est peut-être que Schmidt a en fait reconnu que dans certains cas, le géant de la recherche est obligé de fournir vos données personnelles.

\enquote{Si vous avez vraiment besoin de ce genre de vie privée, soyez conscients que les moteurs de recherche --- dont Google --- conservent les informations un certain temps et qu'il est à noter, par exemple, que nous sommes tous soumis aux États-Unis au Patriot Act et qu'il est donc possible que toutes ces informations soient mises à la disposition des autorités.}

Il y a aussi la possibilité des citations à comparaîtres [des requêtes d'un juge]. Et les piratages informatiques. Mais si ce genre de choses vous dérangent, vous devriez avoir honte de vous, selon Eric Schmidt.
