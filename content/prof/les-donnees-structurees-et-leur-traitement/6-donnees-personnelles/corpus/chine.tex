\section{La Chine commence déjà à mettre en place son système de notation des citoyens prévu pour 2020}
\source{Elsa Trujillo, le Figaro, 27/12/2017\footnote{\url{https://www.lefigaro.fr/secteur/high-tech/2017/12/27/32001-20171227ARTFIG00197-la-chine-met-en-place-un-systeme-de-notation-de-ses-citoyens-pour-2020.php}}.}

Lancé en 2014, le projet vise à récompenser les bons comportements et à punir les mauvais via un système de points. La mise en place a déjà commencé : dès le 1er mai 2018, les Chinois ayant une mauvaise \enquote{note sociale} se verront interdire l'achat de billets de train ou d'avion pour une période pouvant aller jusqu'à un an, a fait savoir Pékin vendredi dernier.

Des points en plus pour l'achat de produits chinois, de bonnes performances au travail ou la publication sur un réseau social d'un article vantant les mérites de l'économie nationale. Des points en moins en cas d'opinions politiques dissidentes, de recherches en ligne suspectes ou de passages piétons traversés à la hâte, alors que le feu est rouge. La Chine travaille depuis 2014 sur un système d'évaluation de ses propres citoyens programmé pour être mis en place en 2020. L'empire du Milieu vient même d'accélérer le calendrier : dès le 1er mai prochain, les individus ayant une mauvaise \enquote{note sociale} seront inscrits sur une liste noire les empêchant d'acheter des billets de train ou d'avion pour une période pouvant aller jusqu'à un an, selon deux communiqués de la Commission nationale de développement de la réforme en date du deux mars et publiés sur internet vendredi dernier.

[…]

D'après [la chercheuse Katika Kühnreich], un tel système fonctionnera en exploitant les mécanismes du jeu, tels que les scores et la comparaison entre amis, pour devenir un insidieux mais très puissant instrument de contrôle social. […] \enquote{Le SCS (pour Social Credit System) utilisera de vrais noms, des données de consommateurs, notamment via Alipay, le système de paiement d'Alibaba, ou des applications de rencontres, dont Baihe}, précise Katika Kühnreich. Les enregistrements des tribunaux, de la police, des banques, des impôts et des employeurs, seront eux aussi utilisés.

En résultera une note globale, à la manière de l'indice de \enquote{désirabilité} attribué par l'application de rencontres Tinder. De cette même note pourra dépendre l'accès des Chinois aux transports publics, à certains services d'État, logements sociaux et formalités de prêts. Katika Kühnreich note que l'accès des plus méritants à certains emplois ainsi que la limitation de l'accès Internet pour les moins performants sont déjà évoqués. Le gouvernement chinois y voit un moyen de mieux contrôler sa population gigantesque en améliorant l'application des règles sur son territoire.
