\section{On a tous des choses à cacher}
\source{Maurits Martijn et Rob Winjnberg, publié le 21 octobre 2013 dans Correspondent, traduit par Courrier International dans le \no1285 du 18 au 24 juin 2015.}

\subsubsection{2. Les relations sociales nécessitent que l'on cache certaines choses}
Chaque être humain montre une personnalité différente en fonction du contexte, du moment, et des personnes qui l'entourent. […]
%Ceux qui disent ``je n'ai rien à cacher'' oublient cette réalité sociale.
Dans un contexte social, tout le monde a des choses à cacher. Au travail, il peut s'agir du caractère colérique qui ressort parfois en présence de son conjoint, qui n'a par contre jamais vu le masque de séducteur que l'on revêt devant des inconnus.

\subsubsection{3. On cache déjà des choses sans s'en rendre compte}
Cacher des choses est devenu tellement naturel qu'on le fait sans réfléchir. Nous cachons notre corps, nos défauts, nos courriels, notre fiche de paie, etc. […]
%Ceux qui affirment ``ne rien avoir à cacher'' passent outre à cette réalité.

\subsubsection{4. L'interdit dépend du contexte}
Dans notre pays [les Pays-Bas] l'adultère n'est pas interdit par la loi, mais en Arabie Saoudite il est passible de la peine de mort. La question de savoir si quelque chose est interdit (et s'il vaut mieux le cacher) dépend donc du contexte. […] Et, quant à votre homosexualité, qui n'intéresse personne [aux Pays-Bas], elle peut vous causer des ennuis devant la douane russe. Bref, on ne peut jamais être sûr de ne pas enfreindre un ``interdit''.

\subsubsection{5. Les interdits évoluent}
Dans les années 1970, des voix s'élevaient publiquement pour demander le droit d'avoir des relations sexuelles avec des enfants. Aujourd'hui, ces mêmes personnes seraient inculpées. On peut donc penser ne rien faire d'``interdit'' (et donc ne rien avoir à cacher) alors que nos actions pourraient nous causer de gros ennuis ultérieurement.

\subsubsection{7. Des faits et gestes mal interprétés}
Imaginons que vous êtes journaliste et que vous voulez écrire un article sur les jeunes Néerlandais qui vont faire le djihad en Syrie. Après de nombreuses recherches sur Internet, vous avez trouvé plusieurs numéros de téléphone et adresses mél pour faire des interviews. Un fois l'article terminé, vous voulez aller en vacances à New York et ne prenez qu'un aller simple parce que vous ne savez pas combien de temps vous voudrez y rester. Avant de partir, vous réalisez que vous et votre petite amie venez d'acheter ensemble votre première maison et qu'elle se retrouverait avec des mensualités trop lourdes s'il vous arrivait quelque chose. Vous prenez donc une assurance-vie juste avant votre départ. Vous avez donc les numéros de plusieurs djihadistes dans votre téléphone, vous avez acheté un aller simple pour New York et vous venez de contracter une assurance-vie. La NSA ne risque-t-elle pas de trouver votre voyage suspect ? Qu'est-ce qui garantit que ses algorithmes de recherche ou les personnes qui les interprètent ne tireront pas de mauvaises conclusions vous concernant ?

\subsubsection{9. On ne peut pas faire confiance aux pouvoirs publics}
Plusieurs fois déjà, les pouvoirs publics néerlandais se sont révélés incapables de bien protéger les données personnelles des citoyens. Quasiment tous les grands projets informatiques de ces dernières années (comme le dossier médical électronique, le vote en ligne ou la carte de transports publics) avaient des failles techniques mettant en danger la sécurité des données. Certes, nous n'avons probablement rien à cacher à l'organisme auquel nous donnes nos informations personnelles, mais cela pose un problème dès lors que d'autres parties ont potentiellement accès à ces données.

\subsubsection{10. Les pouvoirs publics font des erreurs}
Les exemples ne manquent pas. Prenons l'homme d'affaires néerlandais Ron Kowsoleea. Il a été victime de données mal enregistrées dans les bases de données de différents services de recherche néerlandais pendant plus de quinze ans. Kowsoleea a été arrêté et inculpé plusieurs fois parce que quelqu'un d'autre commettait des délits sous son nom et que les pouvoirs publics néerlandais n'étaient pas capables de corriger les erreurs.
Autre exemple : des centaines de travailleurs indépendants ont atterri sur une liste de fraudeurs aux allocations à cause d'un problème informatique. Ils ont été condamnés et ont dû payer de fortes amendes.
%[…]

\subsubsection{11. Quid de nos futurs dirigeants ?}
Si certaines données ont déjà été collectées pour une raison x, pourquoi ne pas les utiliser pour une raison y ? Les spécialistes appellent cela le détournement d'usage. Prenons par exemples les empreintes digitales enregistrées sur les passeports de l'Union européenne. À la base, cette mesure de Bruxelles avait pour but de prévenir l'usurpation d'identité, mais aujourd'hui les empreintes digitales sont également utilisées pour des enquêtes judiciaires. Certes, aujourd'hui, les autorités néerlandaises ne s'y intéressent pas, mas il est naïf de croire qu'il pourrait en être autrement dans l'avenir. Nous ne savons pas encore qui seront les futurs dirigeants et quels seront leurs agendas politiques.
