\section{Ligue 1 : La reconnaissance faciale arrivera-t-elle demain dans les stades de foot ?}
\source{Nicolas Camus, 20 minutes, 24/01/2020\footnote{\url{https://www.20minutes.fr/sport/2702191-20200124-ligue-1-reconnaissance-faciale-arrivera-demain-stades-foot}}.}

Le club [FC Metz] réfléchit au moyen de contrôler [à l'entrée du stade] les personnes sous le coup d'une interdiction commerciale de stade. [Ces interdictions] peuvent être décidées par un club, de manière unilatérale, au motif de \enquote{non-respect des dispositions des conditions générales de vente ou du règlement intérieur du stade relatives à la sécurité des manifestations}.

Mais la personne sanctionnée ne va pas pointer au commissariat, comme pour les deux autres interdictions. C'est au club de la repérer si elle essaie d'entrer. \enquote{C'est impossible pour nos stadiers, qui voient défiler des milliers de personnes, reprend Hélène Schrub. Donc on cherche comment faire appliquer ces interdictions. Quand Two-I (la start-up en question, spécialisée dans l'analyse de flux vidéo) est venue nous présenter cette solution, on s'est dit pourquoi pas} 

Concrètement, des caméras filmeraient les entrées et seraient reliées à un fichier contenant les photos des personnes concernées. Et \enquote{uniquement} elles, insiste la dirigeante. \enquote{En aucun cas nous aurons un fichier avec tous nos abonnés, tous nos clients ou pire encore, toutes les personnes qui entrent un jour au stade. Ça, c'est vraiment de la science-fiction, affirme-t-elle. Je comprends la peur et l'angoisse de certaines personnes, qui peuvent se dire que le club saura le détail de leurs déplacements dans le stade. Mais pas du tout. La base de données du logiciel ne sera alimentée que par des gens interdits de stade}.
