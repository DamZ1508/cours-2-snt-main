\section{Victimes du Stic}
\source{François Koch, L'Express, 19/01/2009\footnote{\url{https://www.lexpress.fr/actualite/societe/victimes-du-stic_732755.html}}.}

Tous les cinq ont souffert du Stic, le fichier des ``infractions constatées'', dont l'utilisation est dénoncée par la Commission nationale informatique et libertés. Témoignages.

[…]

\subsection{Pierre-Alexis : ``Empêché de devenir gendarme''}

``Le 13 novembre 2003, je vois ma petite amie frappée par son père, alors que je venais la chercher. Il se jette sur moi. Je réplique. Je porte plainte, car ma voiture est cabossée. Le lieutenant de police chargé de l'enquête, à Vernon (Eure), affirme que j'ai moi-même abîmé ma voiture ! La justice classe l'affaire sans suite. Je rêvais d'entrer dans la gendarmerie. Je réussis alors le concours. Mais quatre jours avant de l'intégrer, à Rouen (Seine-Maritime), une employée me dit que tout est annulé en raison de 'problèmes administratifs'. Je la supplie de m'en dire plus. Elle me précise, sous le sceau du secret, que mon nom est référencé dans le Stic [comme auteur de dénonciation calomnieuse]. Je suis en dépression depuis plusieurs années et toujours au chômage.'' 

\subsection{Catherine : ``Bloquée pour la magistrature''}

``Avocate depuis douze années, je pose ma candidature pour devenir magistrate. Je suis alors reçue par un agent de police judiciaire, chargé d'enquêter sur moi : ``Avez vous eu des contacts avec la police ?'' me demande-t-il. […] ``Au milieu des années 1990, j'ai été entendu par un policier de la Brigade financière, pour expliquer le concours du cabinet d'avocats, où je travaillais, au montage d'une opération immobilière.'' ``La mémoire vous revient !'', indique le policier refusant toute autre précision. Après avoir reçu une réponse négative à ma candidature à la magistrature, j'écris à la Cnil pour savoir si je suis fichée au Stic. Alors que rien ne m'a jamais été reproché, je découvre avec effroi que je suis inscrite comme ``mise en cause pour escroquerie''. […]''
