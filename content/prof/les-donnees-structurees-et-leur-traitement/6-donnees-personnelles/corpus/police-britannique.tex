\section{Des policiers britanniques détournent les fichiers à des fins personnelles}
\source{Florent Bascoul, Le Monde, 06/07/2016\footnote{\url{https://www.lemonde.fr/pixels/article/2016/07/06/des-policiers-britanniques-detournent-les-fichiers-a-des-fins-personnelles_4965049_4408996.html}}.}

Big Brother Watch, un groupe britannique de défense des libertés publiques contre la surveillance généralisée, vient de publier un rapport pointant le détournement des fichiers électroniques par des policiers à des fins personnelles. Outre-Manche, près de 800 gardiens de la paix ont accédé à des données confidentielles pour leur propre intérêt au cours de ces 5 dernières années.

[…] [Un rapport de Big Brother Watch] donne quelques exemples de l'utilisation détournée de ces fichiers :

\begin{itemize}
  \item Un officier qui trouvait le nom d'une victime amusant a pris une photo de son permis de conduire pour l'envoyer à un ami sur Snapchat. L'officier a démissionné pendant la procédure disciplinaire qui l'a visé ;
  \item Un policier a été licencié après avoir photographié et diffusé des données sensibles contre rémunération ;
  \item Un agent de police a été renvoyé après avoir transmis des informations confidentielles à un membre de sa famille à propos d'un détenu ;
  \item Un gardien de la paix a jugé bon d'informer un ancien collègue que l'un de ses voisins avait été condamné pour agression sexuelle. Le rapport ne précise pas si le policier a été puni.
\end{itemize}

L'association déplore l'impunité des auteurs des détournements données sensibles. Selon les chiffres communiqués, seuls 3 \% des policiers ont été poursuivis en justice ou condamnés. Dans la plupart des cas, aucune action disciplinaire n'a été engagée.
