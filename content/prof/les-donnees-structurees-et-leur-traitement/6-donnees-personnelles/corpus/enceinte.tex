\section{Données persos : Européens, lisez bien la petite histoire de ce père américain}
\source{Xavier de la Porte, L'Obs, 18/11/2016\footnote{\url{https://www.nouvelobs.com/rue89/rue89-monde/20140311.RUE2598/donnees-persos-europeens-lisez-bien-la-petite-histoire-de-ce-pere-americain.html}}.}

[Je] vais vous raconter une histoire. Elle s'est déroulée il y a deux ans aux Etats-Unis, dans la banlieue de Minneapolis. Un homme en colère demande à voir le directeur de sa grande surface habituelle, un Target (une chaîne de grands magasins qui, vous allez voir porte merveilleusement son nom). Il est très énervé parce sa fille, qui a 16 ans, qui est encore au lycée, reçoit des publicités provenant de Target lui vantant des habits de bébé et des couches. \enquote{Vous voulez la pousser à tomber enceinte ?} demande-t-il au directeur, qui ne sait pas bien quoi répondre, et qui est gêné au point que deux jours plus tard, il appelle l'homme pour s'excuser à nouveau.

Sauf que cette fois-ci, c'est l'homme qui s'excuse : \enquote{J'ai parlé avec ma fille. Il se passait chez moi des choses dont je n'étais pas au courant, elle est enceinte. C'est pour août.}

Tout cela a donné lieu à une vaste enquête sur les méthodes de la chaîne Target. Où on s'est aperçu que la chaîne avait un système de publicité ultraciblée, le document papier envoyé au domicile pouvant être personnalisé quasiment à l'unité. Où on s'est aperçu que cette publicité ultraciblée était le fruit d'un travail très précis et très savant de récolte de données et de travail de données, un travail dû à un jeune statisticien du nom de Andrew Pole.

Le principe est simple : en faisant vos courses, vous donnez un nombre incalculable d'informations sur vous-mêmes, en prenant une carte de fidélité vous permettez qu'elles soient associées à un nom, à une adresse. Et le tour est joué. Ensuite, il suffit de faire un gros travail statistique, de construire les algorithmes qui font le lien entre des habitudes d'achat, l'évolution dans ces habitudes et des changements dans la vie (le fait d'avoir un gigantesque corpus permet de faire des liens de corrélation très fins, entre par exemple le changement de type de savon acheté et la grossesse, le fait de ne plus acheter tel type d'aliment et une maladie, etc.). Ainsi, le magasin sait du client ce que ses proches peuvent ignorer.
