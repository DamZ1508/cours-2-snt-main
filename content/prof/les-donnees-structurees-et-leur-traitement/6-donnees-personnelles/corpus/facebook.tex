\section{Pour le fondateur de Facebook, la protection de la vie privée n'est plus la norme}
\source{Le Monde, 11 janvier 2010\footnote{\url{https://www.lemonde.fr/technologies/article/2010/01/11/pour-le-fondateur-de-facebook-la-protection-de-la-vie-privee-n-est-plus-la-norme_1289944_651865.html}}.}

``Les gens sont désormais à l'aise avec l'idée de partager plus d'informations différentes, de manière plus ouverte et avec plus d'internautes. (...) La norme sociale a évolué.'' Le PDG de Facebook, Mark Zuckerberg, est revenu dimanche, à San Francisco, sur la modification des paramètres de vie privée de son réseau social, et estimé que les 350 millions d'utilisateurs du site n'attachent plus autant d'importance à la protection de leurs données personnelles.

Pour le fondateur du plus grand réseau social au monde, cette évolution justifie les modifications des paramètres de vie privée du site, mises en place mi-décembre, et vivement critiquées par les associations de défense de la vie privée. Mark Zuckerberg, qui s'était lui-même fait piéger par le changement de paramètres sur son propre compte personnel, a estimé que cette évolution du site était nécessaire, et reflétait ``ce que seraient les normes si nous lancions le site aujourd'hui''. […]

Pour Mark Zuckerberg, ce sont principalement les jeunes générations qui ont une notion différente de ce qu'est la vie privée, et de la manière dont elle doit être protégée.
