\section{Scandale Facebook-Cambridge Analytica}
\source{Wikipédia, consultée le 28/02/2020\footnote{\url{https://fr.wikipedia.org/wiki/Scandale_Facebook-Cambridge_Analytica}}}

Le scandale Facebook-Cambridge Analytica […] renvoie aux données personnelles de 87 millions d'utilisateurs Facebook que la société Cambridge Analytica (CA) a commencé à recueillir dès 2014. Ces informations ont servi à influencer les intentions de votes en faveur d'hommes politiques qui ont retenu les services de CA. […]

En juillet 2015, l'implication de CA dans les primaires présidentielles du Parti républicain américain de 2016 est dévoilée. En décembre 2015, le journal The Guardian rapporte que l'homme politique américain Ted Cruz a utilisé les données de CA, les personnes visées ignorant que des sociétés exploitaient ces informations. CA aurait participé en 2016 à la campagne électorale de Donald Trump.

En mars 2018, The New York Times, The Guardian et Channel 4 News rapportent plus de détails sur la fuite de données grâce aux révélations de l'ancien salarié de Cambridge Analytica Christopher Wylie, qui a fourni des éclaircissements sur la taille de la fuite, la nature des données personnelles et les échanges entre Facebook, Cambridge Analytica et des personnalités politiques qui avaient retenu les services de CA dans le but d'influencer les intentions de votes. Selon Christopher Wylie : \enquote{Sans Cambridge Analytica, il n'y aurait pas eu de Brexit} 
