\section{Lettre ouverte à ceux qui n'ont rien à cacher}
\source{Jean-Marc Manach, Internetactu.net, 21/05/2010\footnote{\url{http://www.internetactu.net/2010/05/21/lettre-ouverte-a-ceux-qui-nont-rien-a-cacher/}}.}

Il existe de très nombreuses façons d'attenter à la vie privée de quelqu'un, et que même ceux qui n'ont \enquote{rien à cacher} peuvent en faire les frais.

Les milliers de Français nés à l'étranger qui, l'an passé, ont connu les pires difficultés pour renouveler leurs papiers, parce que suspectés de fraudes aux titres d'identité par des fonctionnaires tatillons ou suspicieux, devant leur rapporter moult papiers et preuves de filiation et de nationalité, n'avaient rien à cacher.

Ce SDF qui s'est vu refuser le renouvèlement de son RSA, au motif qu'il était trop propre, tout comme cette mère de famille qui a connu pareille mésaventure parce qu'on la soupçonnait de ne plus être célibataire, et qui dut faire le tour de ses voisins pour leur demander de témoigner qu'aucun homme ne vivait chez elle (la contrôleuse de la CAF vint fouiller ses tiroirs en lui demandant à qui appartenait les petites culottes), n'avaient eux non plus rien à se reprocher.

Branly Nsingi, un Congolais de 21 ans résidant en France, parti en vacances en Côte d'Ivoire et qui y est décédé d'une crise cardiaque après que les autorités lui aient refusé de rentrer à Paris parce que son passeport n'était pas biométrique (le même avait pourtant été validé au départ), ou encore ces 32 Marocains placés en rétention, et expulsés, alors qu'ils… rentraient tranquillement chez eux, n'avaient rien fait de mal, ce qui ne les a pas empêchés d'être pris dans la nasse de cette société de surveillance et de son usine à gaz sécuritaire qui renversent la charge de la preuve.

Dans le meilleur des mondes, policiers et gendarmes ne feraient jamais de fautes de frappe au moment de saisir le nom d'un suspect, et de ce dont il a été suspecté, dans leurs fichiers de suspects. Dans les faits, nombreuses sont les victimes qui sont fichées comme suspectes, sans parler des problèmes d'homonymie, d'absences de mises à jour des fichiers, de détournements de ces fichiers… En 2008, la CNIL a ainsi recensé 83 \% d'erreurs dans les fichiers policiers qu'elle a été amenés à contrôler.

Dans le meilleur des mondes, ceux qui sont payés pour regarder, toute la journée, les écrans de contrôle des caméras de vidéosurveillance, ne feraient jamais de délit de faciès, et ne se permettraient jamais de zoomer sur les décollettés de ces dames. Dans les faits, \enquote{15 \% du temps passé par les opérateurs devant leurs écrans de contrôle relèverait du voyeurisme, 68 \% des noirs qui sont surveillés le sont sans raison spéciale, tout comme 86 \% des jeunes de moins de 30 ans, et 93 \% des hommes}.
