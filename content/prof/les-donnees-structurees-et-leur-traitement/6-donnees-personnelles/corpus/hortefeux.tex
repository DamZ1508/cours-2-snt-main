\section{Brice Hortefeux --- Déclaration sur le site du ministère de l'intérieur}
\source{Brice Hortefeux (alors ministre de l'intérieur), septembre 2009, sur le site du ministère de l'intérieur. Cité par Jean-Marc Manach, \emph{Hortefeux fustige la vidéosurveillance dont il a fait l'objet}\footnote{\url{https://www.lemonde.fr/blog/bugbrother/2009/09/15/hortefeux-fustige-la-videosurveillance-dont-il-a-fait-lobjet/}}.}

\bigskip

Je suis naturellement attaché à la préservation des libertés individuelles. Je le dis clairement, et chacun peut le voir, la vidéo, c'est de la protection avant d'être de la surveillance. Les caméras ne sont pas intrusives, elles ne sont pas là pour épier, mais pour protéger.

Vous le savez, les caméras de protection font déjà partie de notre quotidien : lorsque vous faites vos courses au supermarché, lorsque vous retirez de l'argent au guichet de votre banque ou que vous utilisez les transports en commun, vous êtes filmés, vous le savez déjà. Qui cela dérange t-il ?

Si vous n'avez rien à vous reprocher, vous n'avez pas à avoir peur d'être filmés ! Instaurer la vidéo-protection, c'est identifier les fauteurs de troubles, c'est décourager les délinquants ; c'est, surtout, veiller sur les honnêtes gens. 
