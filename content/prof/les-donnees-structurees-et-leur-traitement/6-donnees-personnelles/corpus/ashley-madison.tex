\subsection*{Pourquoi il ne faut pas rire du hack d'Ashley Madison}
\source{Kevin Poireault, Les Inrockuptibles, 20/08/2015\footnote{\url{https://www.lesinrocks.com/2015/08/20/actualite/actualite/pourquoi-il-ne-faut-pas-rire-du-hack-dashley-madison/}}.}

Si la révélations des données du site de rencontre pour infidèles Ashley Madison a d'abord fait rire beaucoup d'internautes, de plus en plus de gens s'inquiètent aujourd'hui de l'impact que ce piratage pourrait avoir sur l'avenir de la vie privée en ligne.

Mardi 18 août, The Impact Team a tenu parole. Comme aucun des deux sites d'Avid Media Life n'a été fermé, les hacktivistes ont publié les données de près de 35 millions d'utilisateurs d'Ashley Madison (sur 40 millions d'utilisateurs au total, selon le site lui-même). Il y a un mois, ce groupe de hackers a réussi à pirater le site de rencontres adultères AshleyMadison.com et à s'emparer des données de ses utilisateurs. Leur objectif : obliger Avid Media Life (le propriétaire du site), à le fermer ainsi qu'EstablishedMen.com (qui propose aux jeunes filles de trouver un suggar daddy\footnote{\emph{Suggar daddy :} Homme qui entretient une femme plus jeune que lui en échange d'une relation amoureuse.}). 

En divulguant ces données, les hackers ne revendiquent pas seulement un acte moral - moraliste pour certains - mais visent aussi à dénoncer la fragilité du système de sécurité d'Avid Media Life, qui aurait ``menti'' à ses utilisateurs en leur affirmant que la confidentialité d'Ashley Madison était assurée. Ce faisant, ces ``lanceurs d'alerte'' ont compliqué - et parfois même mis en danger - la vie de ces internautes.

En tout, 9,7 gigabytes de données ont été divulgués sur le darknet --- un réseau accessible uniquement via des plateformes spéciales, comme le navigateur Tor. On y trouve les coordonnées géographiques et bancaires (ou du moins une partie), les transactions bancaires des sept dernières années ainsi que les identifiants de chaque utilisateur.

[…]

Sur Twitter, les premières réactions amusées ont aujourd'hui laissé place à ces mêmes avertissements, de la part de spécialistes de la vie privée sur Internet mais aussi d'utilisateurs ordinaires. Comme cette twitto, qui publie, à destination de ceux qui ``essaient encore de justifier le leak d'Ashley Madison'', le témoignage d'un utilisateur saoudien et gay dont les données ont été révélées et qui exprime sa peur d'être condamné à mort dans son pays sur le réseau social Reddit.

Sur le message on peut lire que ce jeune homme a fréquenté le site Ashley Madison lorsqu'il était aux Etats-Unis et que, aujourd'hui rentré en Arabie Saoudite, il craint pour sa vie car ses données ont été divulguées sur la Toile. Il
[demande] % affirme avoir réuni presque suffisamment d'argent pour se payer un billet d'avion et  demande à la communauté de Reddit
comment il peut s'y prendre pour acquérir le statut de réfugié.
