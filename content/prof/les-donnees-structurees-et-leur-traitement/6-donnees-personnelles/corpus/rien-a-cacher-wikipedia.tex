\section{Rien à cacher (argument)}
\source{Wikipédia, page consultée le 28/02/2020\footnote{\url{https://fr.wikipedia.org/wiki/Rien_à_cacher_(argument)}}.}

\subsection{Mauvais usage des données collectées}

Le professeur de droit Daniel J. Solove a exprimé son opposition au \enquote{rien à cacher} : il affirme qu'un gouvernement, et par extension toute organisation collectant des données, peut diffuser sur une personne des informations susceptibles de lui nuire, ou bien utiliser des informations la concernant pour lui refuser l'accès à certains services, même si cette personne n'a commis en pratique aucune mauvaise action. […] Il n'est pas exclu en effet que des données soient diffusées par mégarde, ou bien que quelqu'un obtienne un accès frauduleux à leur support de stockage.

Mais plus généralement encore se pose la question de la réutilisation des données. Lorsque celles-ci ne sont pas collectées de façon transparente, est-il possible de s'assurer qu'il n'y aura pas de dérives ?

[…]

Un autre argument important, et qui découle du précédent, est la persistance des données collectées dans le temps : rien ne garantit que leur usage sera toujours le même que celui qui aura été annoncé initialement. Par exemple, le cas des Juifs d'Allemagne, qui sont allés se déclarer en 1936, avant que l'ampleur du projet de leur répression ne soit dévoilée. Rien ne garantit qu'un comportement toléré aujourd'hui le soit toujours dans le futur.

Enfin, les données personnelles peuvent également être utilisées et analysées par des sociétés privées pour adapter leur offre en fonction du profil qu'ils auront dressé. C'est d'ailleurs ce que met en avant l'analyste Klara Weiand dans le documentaire Nothing to Hide :

\enquote{Une assurance pourrait devenir plus chère pour vous, le prix de marchandises pourrait également varier en fonction de votre revenu estimé et de votre propension à acheter ces produits}.
