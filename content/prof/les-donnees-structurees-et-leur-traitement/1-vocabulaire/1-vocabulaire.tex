%%%%%%%%%%%%%%%%%%%%%%%%%%%%%%%%%%%%%%%%%%%%%%%%%%%%%%%%%%%%%%%%%%%%%%%%%%%%%%%%
% Copyright 2020-2023 Louis Paternault --- http://snt.ababsurdo.fr
%
% Publié sous licence Creative Commons Attribution-ShareAlike 4.0 International (CC BY-SA 4.0)
% http://creativecommons.org/licenses/by-sa/4.0/deed.fr
%%%%%%%%%%%%%%%%%%%%%%%%%%%%%%%%%%%%%%%%%%%%%%%%%%%%%%%%%%%%%%%%%%%%%%%%%%%%%%%%

% Pour compiler :
%$ lualatex $basename

\documentclass[12pt]{article}

\usepackage{calc}
\usepackage[
includehead,
paperwidth=21cm,
paperheight={297mm/3},
margin=5mm,
headsep=3mm,
]{geometry}

\usepackage{textcomp}
\usepackage[francais]{babel}
\usepackage{hyperref}
\hypersetup{
  unicode=true,
  urlcolor=cyan,
  pdfauthor={Louis Paternault},
  pdfproducer={© Louis Paternault — CC-BY-SA-4.0 — http://snt.ababsurdo.fr},
  hidelinks,
}

\usepackage{fancyhdr}
\pagestyle{fancy}
\fancyhead[L]{\textsc{SNT > Les Données structurées et leur Traitement}}
\fancyhead[R]{\textsc{Activité : Un peu de vocabulaire…}}
\fancyfoot[C]{}

\begin{document}

% Cette séance est au départ inspirée de la ressource publiée sur Éduscol : Les Données structurées et leur Traitement — VOCABULAIRE.
% https://cache.media.eduscol.education.fr/file/SNT/93/8/RA19_Lycee_G_SNT_2nd_vocabulaire_donnees_1151938.pdf

\begin{enumerate}
  \item Choisissez un livre, puis décrivez-le en une ligne, afin qu'un camarade puisse l'identifier et aller le retrouver dans une bibliothèque.
    \item On considère le livre dont la notice est projetée au tableau.
    \begin{enumerate}
    \item Quelles données (ou ensemble de données) permettent de l'identifier de manière unique ?
    \item Quelles données sont plutôt destinées aux humains ? Quelles données sont plutôt destinées aux ordinateurs ?
    \item Quels sont les types de données utilisées ?
    \end{enumerate}
  \item On souhaite identifier votre sac à dos parmi une collection d'objet (dans le catalogue d'un grand magasin par exemple).
  Donnez les descripteurs que vous pouvez utiliser, ainsi que leur type, et quelques exemples de valeurs.
  \item On souhaite \emph{vous} identifier parmi l'ensemble de la populaton française.
  Donnez les descripteurs que vous pouvez utiliser, ainsi que leur type, et quelques exemples de valeurs.
  \item Comment peut-on l'ensemble de la collection de la première question (les livres) dans un fichier informatique ?
\end{enumerate}



\end{document}
