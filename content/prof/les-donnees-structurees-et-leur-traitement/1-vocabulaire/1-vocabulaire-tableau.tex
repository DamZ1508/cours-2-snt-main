%%%%%%%%%%%%%%%%%%%%%%%%%%%%%%%%%%%%%%%%%%%%%%%%%%%%%%%%%%%%%%%%%%%%%%%%%%%%%%%%
% Copyright 2023 Louis Paternault --- http://snt.ababsurdo.fr
%
% Publié sous licence Creative Commons Attribution-ShareAlike 4.0 International (CC BY-SA 4.0)
% http://creativecommons.org/licenses/by-sa/4.0/deed.fr
%%%%%%%%%%%%%%%%%%%%%%%%%%%%%%%%%%%%%%%%%%%%%%%%%%%%%%%%%%%%%%%%%%%%%%%%%%%%%%%%

% Pour compiler :
%$ lualatex $basename

\documentclass[14pt, aspectratio=43]{beamer}

\usepackage{2122-pablo}
\usepackage{2122-pablo-beamer}
\usepackage{2122-pablo-paternault}

\title{Les Données structurées et leur Traitement}
\subtitle{Un peu de vocabulaire…}

\begin{document}

\begin{frame}
% D'après https://catalogue.bnf.fr/ark:/12148/cb347156037.public
% Légèrement adapté par mes soins.
\begin{itemize}
\item \textbf{Auteur(s) :} Ésope (0620?-0560? av. J.-C.).
\item \textbf{Titre conventionnel :} Fables (français). Extrait]
\item \textbf{Titre(s) :} Fables 
\item \textbf{Publication :} Paris : Gallimard jeunesse, DL 2021
\item \textbf{Impression :} impr. en Espagne
\item \textbf{Description matérielle :} 1 vol. (100 p.) : ill. ; 18 cm
\item \textbf{Collection :} Folio junior. Textes classiques
\item \textbf{Autre(s) auteur(s) :} Saillard, Rémi (1960-....). Illustrateur ; Chambry, Émile (1864-1951). Traducteur ; Cossou, Bruno. Éditeur scientifique.
\item \textbf{Identifiants :} ISBN 978-2-07-512050-0 (br.) 
\item \textbf{Prix :} 4,50 EUR
\end{itemize}
\end{frame}

\begin{frame}
% D'après https://catalogue.bnf.fr/ark:/12148/cb347156037.public
% Légèrement adapté par mes soins.
\begin{itemize}
%\item \textbf{Type(s) de contenu et mode(s) de consultation :} Texte noté : sans médiation
\item \textbf{Auteur(s) :} Bellonci, Maria (1902-1986) 
\item \textbf{Titre(s) :} Lucrèce Borgia
\item \textbf{Traduction de :} Lucrezia Borgia
\item \textbf{Publication :} Bruxelles : Éditions Complexe ; [Paris] : [diffusion Presses universitaires de France], 1983
\item \textbf{Impression :} impr. en Belgique
\item \textbf{Description matérielle :} IV-469 p. : couv. ill. en coul. ; 23 cm
\item \textbf{Collection :} Le Temps et les hommes ; 14
%\item \textbf{Lien à la collection :} Le Temps et les hommes 
\item \textbf{Note(s) :} Bibliogr. p. 463-469
%\item \textbf{Sujet(s) :} Borgia, Lucrèce (1480-1519) 
\item \textbf{Identifiants, prix et caractéristiques :} ISBN 2-87027-106-9 (Br.) : 125 F
%\item \textbf{Identifiant de la notice  :} ark:/12148/cb347156037
%\item \textbf{Notice n° :}  FRBNF34715603 
\end{itemize}
\end{frame}

\begin{frame}
\frametitle{Cours}
   \begin{itemize}
   \item Une \textbf{collection} est un ensemble d'objets (concrets ou abstraits) dont on collecte des données, partageant les mêmes descripteurs.
   \item Un \textbf{objet} est un élément de cette collection.
   \item Un \textbf{descripteur} désigne l'aspect de l'objet concerné par la donnée.
   \item Une \textbf{valeur} est l'information elle-même.
   \item Le \textbf{type} d'une valeur est la nature de cette information.
   \end{itemize}
   De plus, si une donnée concerne une personne, on dit que c'est une \textbf{donnée personnelle}.
\end{frame}

\begin{frame}
\frametitle{Exemple}
  On s'intéresse aux données gérées par une bibliothèques.
 
 \begin{enumerate}
 \item L'ensemble des usagers de la bibliothèque est une \emph{collection} ; l'ensemble des livres en est une autre.
 \item Chacun des livres est un \emph{objet} de cette collection.
 \end{enumerate}

On s'intéresse au livre \emph{Les Fleurs du mal}, de Charles Baudelaire.
\end{frame}

\end{document}
