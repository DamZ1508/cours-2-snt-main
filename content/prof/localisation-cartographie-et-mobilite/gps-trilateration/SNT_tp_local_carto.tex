\input{preambule_twilo.tex}
\usepackage{tkz-tab}

%Couleur pour l'entète des tableaux
\definecolor{emerald}{rgb}{0.31, 0.78, 0.47}
\usepackage{layout}
%%%%%%%%%%%%%%%%%%%%%%%%%%%%%%%%%%%%%%%%%%%
% *******************EnTètes et Pieds de page *****************************%
%%%%%%%%%%%%%%%%%%%%%%%%%%%%%%%%%%%%%%%%%%%

\pagestyle{fancy}
\renewcommand{\headrulewidth}{0pt}
%\lhead{2 Maths --- ?? - Titre}
%\chead{\thepage/\pageref{LastPage}}
%\rhead{Cours}
\renewcommand{\footrulewidth}{1pt}
\lfoot{2GT --- SNT -- Localisation et Cartographie}
\cfoot{}
\rfoot{\thepage/\pageref{LastPage}}

\begin{document}
	
	%%%%%%%%%%%TITRE DOCUMENT%%%%%%%%%%%%%%%%%%%%%
	
	\begin{titre}{yellow!50!white}{Seconde - SNT}
		\centering
		{\Large \textsc{\textbf{Localisation et Cartographie}}}
	\end{titre}

%%%%% Le contenu %%%%%%%%
	

		Dans ce TP, nous allons imiter manuellement dans des cas simples, simplistes et simplifiés les principes de bases de la géolocalisation et de la navigation numérique.
		\bigskip
		
		Nous travaillerons dans un premier temps sur la carte de France ci-dessous :\bigskip
		
		
		\includegraphics[width=1\linewidth]{img/carte.png}
	
\newpage
	\begin{exersb}
		
	
			Il est exactement midi (on considérera que votre horloge est très précise et à l’heure).\\
			La vitesse de la lumière est de $299~792~458~m/s$.\\
			Vous venez de recevoir les informations suivantes de la part de trois satellites :
			
			\begin{tblr}{hlines,vlines,colspec={*4{X[c,m]}},row{1}={black!30,font=\bfseries},cell{2-6}{1}={black!30,font=\bfseries}}
				&Satellite 1&Satellite 2&Satellite 3\\
				Longitude (DD)&$3,0558^\circ$&$-4,5052^\circ$&$-1,4651^\circ$\\
				Latitude (DD)&$50,634^\circ$&$48,3857^\circ$&$43,4896^\circ$\\
				Altitude (km)&$20~200$&$20~200$&$20~200$\\
				Heure (h:min:s)&$11:59:59,93257$&$11:59:59,932564$&$11:59:59,932604$\\
			\end{tblr}
		
		\begin{enumerate}
			\item Positionnez les satellites sur les villes correspondantes.

			\item Expliquez comment a été calculée la première colonne du tableau ci-dessous puis complétez les deux dernières colonnes de la même manière.\medskip
				
				\begin{tblr}{hlines,vlines,colspec={Q[c,m]*3{X[c,m]}},row{1}={black!30,font=\bfseries},cell{2-7}{1}={black!30,font=\bfseries}}
					&Satellite 1&Satellite 2&Satellite 3\\
					Altitude (km)&$20~200$&$20~200$&$20~200$\\
					Heure (h:min:s)&$11:59:59,93257$&$11:59:59,932564$&$11:59:59,932604$\\
					Délai (s)&$0,06743$&&\\
					Distance réelle (km)&$20~215$&&\\
					Distance au sol (km)&$779$&&\\
					Distance sur la carte (cm)&$11,7$&&\\					
				\end{tblr}
				\medskip
				
%%% le corrigé
%			 \begin{tblr}{hlines,vlines,colspec={Q[c,m]*3{X[c,m]}},row{1}={black!30,font=\bfseries},cell{2-7}{1}={black!30,font=\bfseries}}
%					&Satellite 1&Satellite 2&Satellite 3\\
%					Altitude (km)&$20~200$&$20~200$&$20~200$\\
%					Heure (h:min:s)&$11:59:59,93257$&$11:59:59,932564$&$11:59:59,93257$\\
%					Délai (s)&$0,06743$&$0,067436$&$0,067396$\\
%					Distance réelle (km)&$20~215$&$20~217$&$20~205$\\
%					Distance au sol (km)&$779$&$829$&$449$\\
%					Distance sur la carte (cm)&$11,7$&$12,4$&$6,7$\\					
%				\end{tblr}
			
			\item Placez-vous sur la carte.

			\item Pourquoi 3 satellites étaient nécessaires ?
		
			\item Pourquoi peut-on se passer du quatrième satellite ?
		
		\end{enumerate}
	\end{exersb}
	\begin{exersb}

			Le GPS de l’ami que vous devez rejoindre viens d’émettre le NMEA suivant :
			
			\hfill\textbf{GPGGA,123959.301,4851.558,N,00220.529,E,1,12,1.0,0.0,M,0.0,M,,*67}\hfill\allowbreak
		
	 	Positionnez votre ami sur la carte.
			

	\end{exersb}
\newpage
	\begin{exersb}
	
			Afin de rejoindre votre ami, vous avez récupéré et regroupé dans un tableau les temps de trajets (en minutes) séparant différentes villes qui pourraient se situer sur votre parcours.\medskip
			
			\begin{tblr}{hlines,vlines,colspec={Q[c,m]*{12}{X[c,m]}},row{1}={black!30,font=\bfseries},cell{2-13}{1}={black!30,font=\bfseries}}
				&\rotatebox[origin=c]{90}{\parbox{4cm}{\textbf{Avignon}}}&\rotatebox[origin=c]{90}{\parbox{4cm}{\textbf{Carcassonne}}}&\rotatebox[origin=c]{90}{\parbox{4cm}{\textbf{Clermont-Ferrand}}}&\rotatebox[origin=c]{90}{\parbox{4cm}{\textbf{Dijon}}}&\rotatebox[origin=c]{90}{\parbox{4cm}{\textbf{Limoges}}}&\rotatebox[origin=c]{90}{\parbox{4cm}{\textbf{Lyon}}}&\rotatebox[origin=c]{90}{\parbox{4cm}{\textbf{Montpellier}}}&\rotatebox[origin=c]{90}{\parbox{4cm}{\textbf{Nîmes}}}&\rotatebox[origin=c]{90}{\parbox{4cm}{\textbf{Orléans}}}&\rotatebox[origin=c]{90}{\parbox{4cm}{\textbf{Paris}}}&\rotatebox[origin=c]{90}{\parbox{4cm}{\textbf{Saint-Étienne}}}&\rotatebox[origin=c]{90}{\parbox{4cm}{\textbf{Toulouse}}}\\
				Avignon&\SetCell{bg=black!15}&\SetCell{bg=black!15}&\SetCell{bg=black!15}&\SetCell{bg=black!15}&\SetCell{bg=black!15}&\SetCell{bg=black!15}&\SetCell{bg=black!15}&\SetCell{bg=black!15}&\SetCell{bg=black!15}&\SetCell{bg=black!15}&\SetCell{bg=black!15}&\SetCell{bg=black!15}\\
				Carcassonne&&\SetCell{bg=black!15}&\SetCell{bg=black!15}&\SetCell{bg=black!15}&\SetCell{bg=black!15}&\SetCell{bg=black!15}&\SetCell{bg=black!15}&\SetCell{bg=black!15}&\SetCell{bg=black!15}&\SetCell{bg=black!15}&\SetCell{bg=black!15}&\SetCell{bg=black!15}\\
				Clermont-Ferrand&&&\SetCell{bg=black!15}&\SetCell{bg=black!15}&\SetCell{bg=black!15}&\SetCell{bg=black!15}&\SetCell{bg=black!15}&\SetCell{bg=black!15}&\SetCell{bg=black!15}&\SetCell{bg=black!15}&\SetCell{bg=black!15}&\SetCell{bg=black!15}\\
				Dijon&&&&\SetCell{bg=black!15}&\SetCell{bg=black!15}&\SetCell{bg=black!15}&\SetCell{bg=black!15}&\SetCell{bg=black!15}&\SetCell{bg=black!15}&\SetCell{bg=black!15}&\SetCell{bg=black!15}&\SetCell{bg=black!15}\\
				Limoges&&&$150$&&\SetCell{bg=black!15}&\SetCell{bg=black!15}&\SetCell{bg=black!15}&\SetCell{bg=black!15}&\SetCell{bg=black!15}&\SetCell{bg=black!15}&\SetCell{bg=black!15}&\SetCell{bg=black!15}\\
				Lyon&$140$&&$120$&$150$&&\SetCell{bg=black!15}&\SetCell{bg=black!15}&\SetCell{bg=black!15}&\SetCell{bg=black!15}&\SetCell{bg=black!15}&\SetCell{bg=black!15}&\SetCell{bg=black!15}\\
				Montpellier&&$100$&$210$&&&&\SetCell{bg=black!15}&\SetCell{bg=black!15}&\SetCell{bg=black!15}&\SetCell{bg=black!15}&\SetCell{bg=black!15}&\SetCell{bg=black!15}\\
				Nîmes&$30$&&&&&&$60$&\SetCell{bg=black!15}&\SetCell{bg=black!15}&\SetCell{bg=black!15}&\SetCell{bg=black!15}&\SetCell{bg=black!15}\\
				Orléans&&&$180$&&$170$&&&&\SetCell{bg=black!15}&\SetCell{bg=black!15}&\SetCell{bg=black!15}&\SetCell{bg=black!15}\\
				Paris&&&&$230$&&&&&$90$&\SetCell{bg=black!15}&\SetCell{bg=black!15}&\SetCell{bg=black!15}\\
				Saint-Étienne&&&$100$&&&$70$&&&&&\SetCell{bg=black!15}&\SetCell{bg=black!15}\\
				Toulouse&&$60$&&&$180$&&&&&&&\SetCell{bg=black!15}\\
			\end{tblr}
		
		\begin{enumerate}
			\item	Représentez ces informations sous forme d’un graphe pondéré.
				
			\newpage
		
			\item
		
				À l’aide de l’algorithme de Dijkstra, déterminez le trajet le plus court pour rejoindre votre ami.
				
				\begin{tblr}{hlines,vlines,colspec={*{12}{X[c,m]}},row{1}={black!30,font=\bfseries}}
					\rotatebox[origin=c]{90}{\parbox{5cm}{A -- \textbf{Avignon}}}&\rotatebox[origin=c]{90}{\parbox{5cm}{C -- \textbf{Carcassonne}}}&\rotatebox[origin=c]{90}{\parbox{5cm}{CF -- \textbf{Clermont-Ferrand}}}&\rotatebox[origin=c]{90}{\parbox{5cm}{D -- \textbf{Dijon}}}&\rotatebox[origin=c]{90}{\parbox{5cm}{Li -- \textbf{Limoges}}}&\rotatebox[origin=c]{90}{\parbox{5cm}{Ly -- \textbf{Lyon}}}&\rotatebox[origin=c]{90}{\parbox{5cm}{M -- \textbf{Montpellier}}}&\rotatebox[origin=c]{90}{\parbox{5cm}{N -- \textbf{Nîmes}}}&\rotatebox[origin=c]{90}{\parbox{5cm}{O -- \textbf{Orléans}}}&\rotatebox[origin=c]{90}{\parbox{5cm}{P -- \textbf{Paris}}}&\rotatebox[origin=c]{90}{\parbox{5cm}{S -- \textbf{Saint-Étienne}}}&\rotatebox[origin=c]{90}{\parbox{5cm}{T -- \textbf{Toulouse}}}\\
					\\
					\\
					\\
					\\
					\\
					\\
					\\
					\\
					\\
					\\
					\\
					\\
				\end{tblr}
				
		\end{enumerate}
	\end{exersb}
\label{LastPage}
\end{document}