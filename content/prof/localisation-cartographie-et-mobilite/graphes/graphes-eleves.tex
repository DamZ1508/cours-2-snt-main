%%%%%%%%%%%%%%%%%%%%%%%%%%%%%%%%%%%%%%%%%%%%%%%%%%%%%%%%%%%%%%%%%%%%%%%%%%%%%%%%
% Copyright 2022-2023 Louis Paternault --- http://snt.ababsurdo.fr
%
% Publié sous licence Creative Commons Attribution-ShareAlike 4.0 International (CC BY-SA 4.0)
% http://creativecommons.org/licenses/by-sa/4.0/deed.fr
%%%%%%%%%%%%%%%%%%%%%%%%%%%%%%%%%%%%%%%%%%%%%%%%%%%%%%%%%%%%%%%%%%%%%%%%%%%%%%%%

% Pour compiler :
%$ lualatex $basename

\documentclass[12pt]{article}

\usepackage{2122-pablo}
\usepackage{2122-pablo-paternault}
\usepackage{2122-pablo-math}

\usepackage[
  a6paper,
  landscape,
  margin=10mm,
  includehead,
]{geometry}
\usepackage{2122-pablo-header}
\fancyhead[L]{\textsc{Snt > Cartographie > Graphes}}

\usepackage{mdframed}

\begin{document}

\begin{mdframed}
  Un réseau de routes peut être représenté par un graphe, où :
  \begin{itemize}
    \item les routes sont représentées par les \blanc{arêtes} ;
    \item les intersections sont représentées par les \blanc{sommets} ;
    \item le graphe peut être \blanc{pondéré} par les distances.
  \end{itemize}

  Il est possible de trouver le plus court chemin entre deux sommets en utilisant \emph{l'algorithme de Dijkstra}.
\end{mdframed}

\begin{exercice*}
  Utiliser l'algorithme de Dijkstra pour déterminer le plus court chemin pour aller d'Annecy à Saint-Étienne.
\end{exercice*}

\end{document}
