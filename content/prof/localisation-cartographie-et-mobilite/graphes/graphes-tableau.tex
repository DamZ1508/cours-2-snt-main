%%%%%%%%%%%%%%%%%%%%%%%%%%%%%%%%%%%%%%%%%%%%%%%%%%%%%%%%%%%%%%%%%%%%%%%%%%%%%%%%
% Copyright 2022-2023 Louis Paternault --- http://snt.ababsurdo.fr
%
% Publié sous licence Creative Commons Attribution-ShareAlike 4.0 International (CC BY-SA 4.0)
% http://creativecommons.org/licenses/by-sa/4.0/deed.fr
%%%%%%%%%%%%%%%%%%%%%%%%%%%%%%%%%%%%%%%%%%%%%%%%%%%%%%%%%%%%%%%%%%%%%%%%%%%%%%%%

% Pour compiler :
%$ lualatex $basename

\documentclass[12pt, aspectratio=43, xcolor=table]{beamer}

\usepackage{2223-pablo}
\usepackage{2223-pablo-beamer}
\usepackage{2223-pablo-paternault}

\title{Cartographie}
\subtitle{Un peu de vocabulaire…}

\begin{document}

\begin{frame}
  Un chauffeur a une livraison à effectuer d'Annecy à Saint-Étienne. Il a accès aux autoroutes suivantes (le prix et la distance de chaque tronçon sont les mêmes pour l'aller et le retour) :
  \begin{center}
    \rowcolors{1}{}{lightgray}
    \begin{tabular}{llcc}
      \toprule
      && Distance (km) & Prix (€) \\
      \midrule
      Annecy & Bourg-en-Bresse & 70 & 25 \\
      Annecy & Chambéry & 50 & 15 \\
      Bourg-en-Bresse & Lyon & 80 & 15 \\
      Chambéry & Grenoble & 60 & 20 \\
      Chambéry & Lyon & 110 & 20 \\
      Grenoble & Lyon & 110 & 15 \\
      Grenoble & Valence & 90 & 15 \\
      Lyon & Saint-Étienne & 60 & 30 \\
      Saint-Étienne & Valence & 120 & 10 \\
      \bottomrule
    \end{tabular}
  \end{center}

  \begin{enumerate}
    \item Quel est l'itinéraire le plus court ?
    \item Quel est l'itinéraire le moins cher ?
  \end{enumerate}

\end{frame}

\end{document}
