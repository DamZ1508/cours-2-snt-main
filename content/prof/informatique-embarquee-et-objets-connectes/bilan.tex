\input{preambule_damien2.tex}     

%%%%%%%%%%%%%%%%%%%%%%%%%%%%%%%%%%%%%%%%%%%
%****************TABLEAUX*******************************%
%%%%%%%%%%%%%%%%%%%%%%%%%%%%%%%%%%%%%%%%%%%



%%%%%%%%%%%%%%%% TABLEAU SIGNE ET VARIATION %%%%%%%%%%%%%%
%\begin{center}
%%\vspace{0.3cm}
%\begin{tikzpicture}
%\tkzTabInit[espcl=2]{$x$ /.8 , $d(x)$ /1.5}{$0$ , $4$,$16$,$24$}
%\tkzTabLine[]{,+,z,-,}
%\tkzTabVar[]{+/ $8$,-/$6$,+/$19$,-/$8$}
%%\tkzTabVal[draw]{1}{2}{0.5}{0}{0}
%\end{tikzpicture}
%\end{center}

%%%%%%%%%%%%%%% TABLEAU %%%%%%%%%%%%%%%%%%%%%%%%%%%%%%
%\renewcommand{\arraystretch}{hauteur}
%\begin{center}
%\begin{tabularx}{0.94\linewidth}{|c|*{7}{>{\centering \arraybackslash}X|}}
%\hline 
%
%\hline 
%\end{tabularx} 
%\end{center}

%%%%%%%%%%%%%%%%%%%%%%%%%%%%%%%%%%%%%%%%%%%
%****************** GRAPHIQUE FCT**********************%
%%%%%%%%%%%%%%%%%%%%%%%%%%%%%%%%%%%%%%%%%%%

%\psset{xunit=0.75cm,yunit=0.75cm,algebraic=true,dimen=middle,dotstyle=o,dotsize=5pt 0,linewidth=1.4pt,arrowsize=2pt 2,arrowinset=0.25}
%\def\xmin{-9} \def\xmax{9} \def\ymin{-10.5} \def\ymax{6.5} \def\dx{0.5} \def\dy{0.5}
%\begin{center}
%		\begin{pspicture*}(\xmin,\ymin)(\xmax,\ymax)
%		\multips(0,\ymin)(0,\dy){35}{\psline[linestyle=dashed,linecap=1,dash=1.5pt 1.5pt,linewidth=0.4pt,linecolor=lightgray]{c-c}(\xmin,0)(\xmax,0)}
%		\multips(\xmin,0)(\dx,0){38}{\psline[linestyle=dashed,linecap=1,dash=1.5pt 1.5pt,linewidth=0.4pt,linecolor=lightgray]{c-c}(0,\ymin)(0,\ymax)}
%		\psaxes[labels=all,labelFontSize=\scriptstyle,labelsep=2pt,xAxis=true,yAxis=true,Dx=1,Dy=1,ticksize=-2pt 0,subticks=2,showorigin=False]{->}(0,0)(\xmin,\ymin)(\xmax,\ymax)[{$x$,0][$y$,90]
%\psclip{%
%\psframe[linestyle=none](\xmin,\ymin)(\xmax,\ymax)}
%		\uput[dl](0,0){\scriptsize $0$}
%		\psplot[linewidth=1.4pt,plotpoints=200, linecolor=blue]{\xmin}{\xmax}{2.79^x}
%}\endpsclip
%		\end{pspicture*}
%\end{center}

%%%%%%%%%%%%%%%%%%%%%%%%%%%%%%%%%%%%%%%%%%%
% *******************MINTED *****************************%
%%%%%%%%%%%%%%%%%%%%%%%%%%%%%%%%%%%%%%%%%%%
\usepackage{minted}

\renewcommand{\theFancyVerbLine}{\textcolor{gray}{\tiny \oldstylenums{\arabic{FancyVerbLine}}}}

\definecolor{bg}{rgb}{0.95,0.95,0.95}

\newminted[python]{python}{fontfamily=tt,linenos=true,autogobble,mathescape=true,python3,fontsize=\small, tabsize=4, samepage=true, rulecolor=gray, numbersep=2pt, bgcolor=bg}

\newmintinline{python}{fontfamily=tt,linenos=true,autogobble,mathescape=true,python3}

\newmintedfile[pythonexternal]{python}{fontfamily=tt,linenos=true,autogobble,mathescape=true,python3}

\newminted[html]{html}{fontfamily=courier, fontsize=\footnotesize, rulecolor=gray, framerule=1.5pt, mathescape=true, texcomments=true, autogobble, tabsize=4, numbersep=8pt}

\newmintinline{html}{fontfamily=courier, fontsize=\small}

\newminted[css]{css}{fontfamily=courier, fontsize=\footnotesize, rulecolor=gray, framerule=1.5pt, mathescape=true, texcomments=true, autogobble, tabsize=4, numbersep=8pt}

\newmintinline{css}{fontfamily=courier, fontsize=\small}

\newminted[algo]{bbcode}{fontfamily=courier, fontsize=\footnotesize, rulecolor=gray, framerule=1.5pt, mathescape=true, texcomments=true, autogobble, linenos=true, tabsize=4, numbersep=8pt}

%%%%%%%%%%%%%%%%%%%%%%%%%%%%%%%%%%%%%%%%%%%
% *******************EnT�tes et Pieds de page *****************************%
%%%%%%%%%%%%%%%%%%%%%%%%%%%%%%%%%%%%%%%%%%%
\pagestyle{fancy}
\setlength{\headheight}{40pt} % Haut de page
\renewcommand{\headrulewidth}{0.8pt}
\setlength{\textheight}{26cm}
\lhead{\em NOM : }
%\chead{\thepage/\pageref{LastPage}}
\rhead{\em Bilan TP2}
\renewcommand{\footrulewidth}{1pt}
\lfoot{2GT --- SNT}
\cfoot{}
%\rfoot{\thepage/\pageref{LastPage}}


%%%%%%%%%%%%%%%%%%%%%%%%%%%%%%%%%%%%%%%%%%%%%%

\begin{document}

\section*{Questions : 2,5 points}

{\black\rule{\linewidth}{2pt}}

%\renewcommand{\arraystretch}{4}
\begin{center}
\begin{tabularx}{0.98\linewidth}{p{16cm}X}
\rowcolor{lightgray} Comment s'appelle le logiciel utilis� pour programmer la carte \texttt{micro:bit} ? & \dots/0,5 \\
\rowcolor{lightgray} & \\
\rowcolor{lightgray} & \\
\rowcolor{lightgray} & \\
�crire la commande permettant d'effacer l'�cran de la carte \texttt{micro:bit}.  & \dots/0,5 \\
 & \\
 & \\
 & \\
\rowcolor{lightgray} �crire la commande permettant d'afficher un c\oe ur. & \dots/0,5 \\
\rowcolor{lightgray} & \\
\rowcolor{lightgray} & \\
\rowcolor{lightgray} & \\
�crire la commande permettant d'attendre une seconde. & \dots/0,5 \\
& \\
& \\
&  \\
\rowcolor{lightgray} �crire la commande permettant de faire d�filer le texte \texttt{Bonjour}. & \dots/0,5 \\
\rowcolor{lightgray} & \\
\rowcolor{lightgray} & \\
\rowcolor{lightgray} & \\
\end{tabularx} 
\end{center}

{\black\rule{\linewidth}{2pt}}

\section*{Programme Compte � rebours : 1,5 points}

\begin{center}
\renewcommand{\arraystretch}{2}
\begin{tabularx}{0.98\linewidth}{p{16cm}X}
\rowcolor{lightgray} Le compte � rebours va de 5 � 0 & \dots/0,5 \\
Le programme attend une seconde apr�s chaque chiffre & \dots/0,5 \\
\rowcolor{lightgray} Le programme fait d�filer \texttt{Partez !} � la fin & \dots/0,5 \\
\end{tabularx} 
\end{center}

\sautpage

\rhead{\em Bilan TP3}
\section*{Questions : 2 points}

{\black\rule{\linewidth}{2pt}}

\begin{center}
\begin{tabularx}{0.98\linewidth}{p{16cm}X}
\rowcolor{lightgray} �crire la ligne permettant de tester si le bouton $A$ est appuy�. & \dots/1 \\
\rowcolor{lightgray} & \\
\rowcolor{lightgray} \texttt{if \dots \dots \dots} & \\
\rowcolor{lightgray} & \\
Compl�ter le test suivant pour qu'il v�rifie si les deux conditions sont vraies en m�me temps. & \dots/1 \\
 & \\
\texttt{if \emph{condition1} \dots \dots \emph{condition2}:} & \\
& \\
\end{tabularx} 
\end{center}

\section*{Programme Compteur : 2 points}

{\black\rule{\linewidth}{2pt}}

\begin{center}
\begin{tabularx}{0.98\linewidth}{p{16cm}X}
\rowcolor{lightgray} Le compteur diminue lorsqu'on appuie sur le bouton $B$. & \dots/1 \\
Le compteur est remis � z�ro lorsqu'on appuie sur les deux boutons en m�me temps. & \dots/1 \\
\end{tabularx} 
\end{center}

\vspace*{2cm}


{\em NOM :} \hspace*{15cm} {\em Bilan TP4}

{\black\rule{\linewidth}{0.8pt}}

\section*{Questions : 2 points}

{\black\rule{\linewidth}{2pt}}

\begin{center}
\begin{tabularx}{0.98\linewidth}{p{16cm}X}
\rowcolor{lightgray} �crire la commande permettant de choisir au hasard et d'afficher un �l�ment de la liste \texttt{LISTE}. & \dots/1 \\
\rowcolor{lightgray} & \\
\rowcolor{lightgray} & \\
\rowcolor{lightgray} & \\
�crire la condition permettant de tester si la carte est secou�e. & \dots/1 \\
 & \\
\texttt{if  \dots \dots \dots} & \\
& \\
\end{tabularx} 
\end{center}

\section*{Programme Choixpeau : 2 points}

{\black\rule{\linewidth}{2pt}}

\begin{center}
\begin{tabularx}{0.98\linewidth}{p{16cm}X}
\rowcolor{lightgray} Une des quatre maisons d�file au hasard quand la carte est secou�e. & \dots/1 \\
  Cela fonctionne plusieurs fois � la suite, en affichant un point d'interrogation en attendant. & \dots/1 \\
\end{tabularx} 
\end{center}

\sautpage

\rhead{\em Bilan TP5}
\section*{Questions : 3 points}

{\black\rule{\linewidth}{2pt}}

\begin{center}
\begin{tabularx}{0.98\linewidth}{p{16cm}X}
\rowcolor{lightgray} �crire la commande permettant de configurer la radio pour utiliser le canal 42. & \dots/1 \\
\rowcolor{lightgray} & \\
\rowcolor{lightgray} & \\
\rowcolor{lightgray} & \\
�crire la commande permettant d'envoyer le message \texttt{"Bonjour"}. & \dots/1 \\
 & \\
 & \\
& \\
\rowcolor{lightgray} �crire la commande permettant de recevoir un message, et de tester si ce message est \texttt{"Bonjour"} (une ou deux lignes). & \dots/1 \\
\rowcolor{lightgray} & \\
\rowcolor{lightgray} & \\
\rowcolor{lightgray} & \\
\end{tabularx} 
\end{center}

\section*{Programme - Bracelet, Clignotant, Station m�t�o : 1 point}

{\black\rule{\linewidth}{2pt}}

\begin{center}
\begin{tabularx}{0.98\linewidth}{p{16cm}X}
Le programme fonctionne comme attendu. & \dots/1 \\
\end{tabularx} 
\end{center}

\vspace*{2cm}


{\em NOM :} \hspace*{15cm} {\em Bilan TP6}

{\black\rule{\linewidth}{0.8pt}}

\section*{Graphe : 2 points}

{\black\rule{\linewidth}{2pt}}

\begin{center}
\begin{tabularx}{0.98\linewidth}{p{16cm}X}
\rowcolor{lightgray} Recopier et compl�ter le graphe repr�sentant la machine � �tats & \dots/2 \\
\rowcolor{lightgray} & \\
\rowcolor{lightgray} & \\
\rowcolor{lightgray} & \\
\rowcolor{lightgray} & \\
\rowcolor{lightgray} & \\
\rowcolor{lightgray} & \\
\end{tabularx} 
\end{center}

\section*{Programme Clignotant : 2 points}

{\black\rule{\linewidth}{2pt}}

\begin{center}
\begin{tabularx}{0.98\linewidth}{p{16cm}X}
\rowcolor{lightgray} Le programme affiche (ou non) les fl�ches correctement selon les boutons appuy�s. & \dots/1 \\
  Les fl�ches clignotent. & \dots/1 \\
\end{tabularx} 
\end{center}
\end{document}